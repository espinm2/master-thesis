\chapter{Pilot Study Results And Analysis} \label{sec:results}

\section{User Base Results}

In the two week timespan that OASIS was publicly available 57 users registered and participated in the pilot user study.
% We are going to have to cite these reddit and facebook
I advertised OASIS on social media outlets such as Facebook and Reddit.
Both of these social media outlets have a wide range of users with varying experiences.
Moreover, one of the online outlets I advertised OASIS on is unofficially affiliated with Rensselaer Polytechnic Institute (RPI).
Figure-\ref{} demonstrates the affiliation of users who registered to use OASIS for this pilot study.
As shown in Figure-\ref{} a majority of participants, that provided their affiliation, are actually not affiliated with RPI. 
This is a big change from previous user studies that only used RPI affiliated participants.
Additionally data shows that majority of participants, who are affiliated with RPI, are undergraduate.
It is particularly interesting that the major of overall participants did not provide information on their affiliation with RPI.
Specifically 65\% of participants did not provide feedback on their affiliation, and not all participants who claimed to be affiliated with RPI specified how they were affiliated;
About 2\% of participants have unknown affiliations with RPI.
The difference between RPI affiliated participants to non-RPI affiliated participants could be a direct results of how and when we advertised OASIS.
Particularly, during this pilot user study we advertised towards non-RPI affiliated outlets first and  RPI affiliated outlets last.
Additionally we asked our participants concerning their experience with architecture and visual arts.
Figure-\ref{} shows the distribution of participants' formal education and job experience in both architecture and visual arts.
A majority of our participants expressed that they have no experience with any of the related fields. 
Also those few participants that do have experience, have had only 1-4 years of exposure to formal architecture education or form visual arts education.
However, there is a user registered on OASIS that claims to have over 10 years of job experience in architecture.

Aside from asking about architecture and visual arts experience, I also let participants elaborate on other relevant experiences.
Some of our participants have had experience in civil engineering, electrical engineering, studio arts, user experience design, and architectural engineering with focus on lighting.
Furthermore, I also asked participants to provide a list of 3D modeling softwares they have had exposure to.
As seen in Figure-\ref{} participants have had the most experience with AudoCad\cite{} and SketchUp{}. 
A few participants have had experience w/ 3dsmax\cite{} and Maya\cite{}.
Again, we let participants elaborate on their experience with other 3D modeling softwares. 
Some other 3D modeling softwares, not shown in Figure-\ref{}, that participants have had experience in include SoilidWorks\cite{}, AGI32\cite{}, Dialux\cite{}, and Daysim\cite{}.
Note that AGI32, Dialux, and Daysim are not specifically 3D modeling software but rather daylight analysis and performance tools.
From data collected on participants' affiliations, experience in related fields, and exposure to modeling software, I can support that OASIS seems accessible to a wide variety of users.

In addition to trying understand if OASIS is accessible, I also wanted feedback on the usability of our sketching interface.
Data on how participants spend time on OASIS can provide insight on user behavior.
Figure-\ref{} illustrates the distribution of participants in relation to their time spend on OASIS.
From Figure-\ref{} is it clear that the majority of users registered and participating in the pilot user study spent no time on the actual interface.
On the other hand, the average time spent per participant is about 12 minutes, excluding those participants that do not spend longer than a minute on OASIS past registration.
Although our user retention rate is low, I suspect that the voluntary nature, anonymity, and absence of renumeration in our pilot user study plays a significant role in the large number of participants who register and do not use OASIS.
Furthermore, we have no data on user participation in similar online user studies targeting similar social media outlets.
Equally important, Figure-\ref{} illustrates user time spend on OASIS per page.
Figure-\ref{} shows that participants spend 36\% of their time on the \textit{Sketch a Room} page.
Next participants spend 23\% of their time on the \textit{Create/Load Model} page of OASIS.
It is important to note that first-time users have the option of viewing a short tutorial video; the 1 minute long tutorial video coupled with the fact that the loading page is the first page users are directed to after logging in could directly contribute to the large portion of time users spend on the loading page.
Surprisingly, the page participants spend the least amount of time is on the \textit{Analyze Simulation} page. 
On the \textit{Analyze Simulation} page user would view daylight renderings of user designed models.
On average participants spent only 8\% of their time analyzing their designed models;
When compared to the 25\% of time participants spend on viewing 3D interpretation of their sketches on the \textit{Generate 3D model} page, the time spend on the \textit{Analyze Simulation} page seems remarkably low.
The difference between time spend on these two pages could stem from the fact that the \textit{Analyze Simulation} page is the final page new woulds would visit when navigating OASIS linearly.
Furthermore all of the temporal data collected on pages in OASIS could be effected by time spend filling out feedback questions, leaving OASIS running in the background as participants focus on other task, and users leaving OASIS before creating a single rendering.









