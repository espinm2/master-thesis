\chapter{Pilot User Study Results And Analysis} \label{sec:results}

\section{Participant Background Feedback}

\begin{figure}[h]
	\centering
	\includegraphics[width=0.8\textwidth]{rpi_affilation}
	\caption{User affiliations of participants on OASIS}
	\label{fig:rpi_affilation}
\end{figure}

In the two week timespan that OASIS was publicly available 57 users registered and participated in our pilot user study.
I recruited participants for our pilot user study on large social media outlets such as Facebook and Reddit.
Both of these social media outlets have a wide range of users with varying experiences.
Not to mention, one of the online outlets I recruited participants from is unofficially affiliated with Rensselaer Polytechnic Institute (RPI). 
Figure-\ref{fig:rpi_affilation} demonstrates the affiliation of participants who registered on OASIS.
As shown in Figure-\ref{fig:rpi_affilation} a majority of participants that provided their affiliation are not affiliated with RPI. 
This is a big change from previous user studies, where all the participants were RPI affiliated.
Additionally, Figure-\ref{fig:rpi_affilation} shows that the majority of participants that are affiliated with Rensselaer Polytechnic Institute, are undergraduate.
It is also interesting that the majority of participants choose not to provide information on their affiliation with RPI.
Specifically, 65\% of participants did not provide feedback on their affiliation, and not all participants who claimed to be affiliated with RPI specified how they were affiliated;
About 2\% of participants have unknown affiliations with RPI.
The difference between RPI affiliated participants and non-RPI affiliated participants could be a direct results of how and when we recruited participants for the pilot user study.
During the pilot user study we advertised towards non-RPI affiliated social media outlets first and  RPI affiliated social media outlets last. 
There was more time recruit non-RPI affiliated participants than RPI affiliated participants. \\

\begin{figure}[h]
	\centering
	\includegraphics[width=0.8\textwidth]{exp_edu}
	\caption{Architecture and visual arts experience of OASIS participants }
	\label{fig:exp_edu}
\end{figure}

Similarly, I asked our participants about their experience with architecture and visual arts.
Figure-\ref{fig:exp_edu} shows the distribution of participants' formal education and job experience in both fields of architecture and visual arts.
A majority of our participants expressed that they have no experience with any of the related fields;
As a result, these participants will be referred to as non-experts.
Also a majority of those participants that have experience, generally have only 1-4 years of exposure to formal architecture education or form visual arts education.
However, there is one user registered on OASIS that claims to have over 10 years of job experience in architecture.
Aside from asking about architecture and visual arts experience, I also let participants elaborate on other relevant experiences. 
Some of our participants have had experience in civil engineering, electrical engineering, studio arts, user experience design, and architectural engineering with focus in lighting.
While our current set of participants does not have much experience with architecture, they do encompass a board range of related fields.\\

\begin{figure}[h]
	\centering
	\includegraphics[width=0.8\textwidth]{software_exp}
	\caption{Participants' experience with 3D Modeling Softwares}
	\label{fig:software_exp}
\end{figure}


Furthermore, I also asked participants to provide a list of 3D modeling software they have had exposure to.
As seen in Figure-\ref{fig:software_exp} participants have had the most experience with AudoCad\cite{} and SketchUp\cite{}. 
A few participants have had experience with 3dsMax\cite{} and Maya\cite{}.
Again, we let participants elaborate on their experience with other 3D modeling software. 
Other 3D modeling software, not shown in Figure-\ref{fig:software_exp}, that participants have had experience using include SolidWorks\cite{}, AGI32\cite{}, Dialux\cite{}, and Daysim\cite{}.
Note that AGI32, Dialux, and Daysim are not specifically 3D modeling tools but rather used for daylight analysis and performance.
From user feedback collected in our pilot user study on participants' affiliations, experience in related fields, and exposure to modeling software, determine that OASIS is accessible to a wide variety of users. \\

\begin{figure}[h]
	\centering
	\includegraphics[width=0.8\textwidth]{time_oasis_hist}
	\caption{Distribution of time spent on OASIS per user}
	\label{fig:time_oasis_hist}
\end{figure}

In addition to trying understand if OASIS is accessible, I also wanted feedback on the usability.
Analyzing data on how participants spend time their on OASIS can provide insight on user behavior.
Figure-\ref{fig:time_oasis_hist} illustrates the distribution of participants in relation to their time spend on OASIS.
From Figure-\ref{fig:time_oasis_hist} is it clear that the majority of users registered and participating in the pilot user study spent no time on the actual interface.
On the other hand, the average time spent per participant is about 12 minutes, excluding participants that do not spend longer than a minute on OASIS past registration.
Although our user retention rate is low, I suspect that the voluntary nature, anonymity, and absence of renumeration in our pilot user study plays a significant role in the large number of participants who register and do not use OASIS.

\begin{figure}[h]
	\centering
	\includegraphics[width=0.8\textwidth]{time_chart}
	\caption{Breakdown of time spent on OASIS pages}
	\label{fig:time_chart}
\end{figure}

Furthermore, we have no data on user participation in similar online user studies targeting similar social media outlets.
Equally important, Figure-\ref{fig:time_chart} illustrates participant time spend on OASIS per page.
Moreover, Figure-\ref{fig:time_chart} shows that participants spend 36\% of their time on the \textit{ Sketch a Room} page.
Next participants spend 23\% of their time on the \textit{Create/Load Model}page.
It is important to note that first-time users have the option of viewing a short tutorial video; 
the 1 minute long tutorial video coupled with redirection to the \textit{Create/Load Model} page after logging in, could directly contribute to the large portion of time participants spend on the \textit{Create/Load model} page.
Surprisingly, the page participants spend the least amount of time is on the \textit{Analyze Simulation} page. 
On the \textit{Analyze Simulation} page user can view daylight renderings of user designed models.
On average participants spent only 8\% of their time analyzing their designed models;
when compared to the 25\% of time participants spend on viewing 3D interpretations on the \textit{Generate 3D model} page, the time spend on the \textit{Analyze Simulation} page seems remarkably low.
The difference between time spend on these two pages could stem from the fact that the \textit{Analyze Simulation} page is the final page new users visit when navigating OASIS linearly.
Furthermore, all of the temporal data collected on pages in OASIS could be effected by time spend writing feedback questions, multi-tasking while leaving OASIS running in the background, and leaving OASIS before visiting all pages.
 
\section{Usability Feedback}

Most of the feedback collected on the usability of OASIS is qualitative.
Quantitative boolean feedback would not fully capture how participants are experiencing our sketching interface;
As a result, I collect qualitative feedback to gain insight into how users perceive OASIS.
Figure-\ref{fig:fun} list feedback collected from 14 participants concerning what participants found fun or interesting in our sketching environment.
Overall, 6 of the participants mentioned that the interface was either fun or easy-to-use.
However, some participants found window placement non-intuitive; 
other participants had difficulty with the limited primitives we provide.
While, we did not explicitly ask what they found difficult difficult in the interface in this specific feedback question, their response will be taken into consideration.
The participant who found window placement difficult states that they tried to ``leave a gap between walls to define where to place windows''.
I suspect that the participant, with this issue, must have skipped the tutorial video or not consulted the any of the help options on OASIS.
By the same token the other participant, concerned with the limited options on OASIS, I speculate is comparing our tool to other more fully featured modeling softwares.
As stated before, we intent for OASIS to be an early design tool for use during the schematic design phase of architecture.
As a result, for the pilot study I did not prioritize our selection of furniture items, but choose a three pieces of furniture found commonly in bedrooms.
Interestingly, our only participant with over 10 years of architectural job experience stated that the sketching interface was ``very simple'' and that that they never encountered modeled that could not be interpreted.
Other participants claim they find specific features interesting, including the furniture items we support, skylights, and the daylight simulations.\\

We also asked participants to provide additional features we could add to our sketching interface to extend the flexibility of OASIS.
The two of the most common features requested by participants are the addition of doors on the sketching interface and a wider variety of furniture items.
Also, some participates desired more control over primitives on the sketching interface. 
Including both drag and drop mechanics on walls after initial placement and the precise manipulation of furniture dimensions.
Additionally, our sole participant with over 10 years or architecture experience suggest we offer control over window heights, ceiling heights, and window finishes.
These features are most commonly found in daylighting analysis software and are features important if I plan to define OASIS as a tool for daylighting analysis.
Interestingly, an unanticipated situation with participants' feedback was discovered when analyzing the feedback for this question.
A few of our 14 participants provided duplicate responses from previously asked questions.
Figure-\ref{fig:features} displays  all responses collected that were not duplicate responses to previous questions. \\
 
In addition to collecting feature request feedback, I also ask participants to describe some designs that they were unable to create due to system limitations.
Figure-\ref{fig:limitation} shows participants feedback on limitations of design in our sketching interface.
The most common design limitation observed was the absence of doors in our sketching interface.
From the feedback collected, it seems that participants assumed that they could not design multi-room sketches because of the lack of doors in the sketching interface.
In actuality, previous user studies have confirmed that the physical sketch interpretation algorithm can handle multi-room designs\cite{}.
Other design limitations participants claimed to face included the lack of light shelves in our interface, the inability to place one piece of furniture on top of another, and unavailability of control over scale.
Again, participants also expressed that our selection of furniture items limited designs. \\

Overall, the main take away from participant feedback concerning design limitations is that doors are essential to communicate to users that our sketching interface supports multi room designs.
On a similar note Figure-\ref{fig:dislike} list out participant feedback regarding disliked elements of our sketching interface.
A common dislike in our sketching interface was the absence of scale.
Currently, we convey scale indirectly though statically sized furniture items, however feedback suggest that we make scale more explicit to users.
Interestingly, a participant expressed dislike with our interface because we do not support keyboard shortcuts for common actions, such as undo.
Other dislikes with our sketching interface include the limited collection of furniture we support, absence of doors, the inability to move walls after initial placement, and the the lack of accurately when selecting a geographical locations for sketches.\\

Lastly, we asked participants if there were any elements in our interface that were hard to use.
Feedback from that question can be seen on Figure-\ref{fig:hard2use}.
Many participant responded to this question with stating nothing was hard to use on OASIS.
However, a few users experienced software bugs with the interface and used this feedback question as a means to report them to us.
Aside from a few fixable software bugs, of which did not impact the entire system, a participant found the redundancy of Raphael FreeTransform handles confusing.
FreeTransform handles are three white circles that are overlaid onto furniture items when clicked in our sketching interface.
One circle appears at the center of the furniture item, and the two other circles are paced perpendicularly some distance away from the furniture item, as illustrated in Figure-\ref{fig:oldvh}F.
As of now, these two perpendicular handles are used solely to rotate items.
Participant feedback help us note overlooked redundancies in our interface such as two rotation FreeTransform handles that perform the same action.\\

To completely analyze the feedback collected from our sketching interface we must understand that omission of feedback could potentially be used to communicate feedback.
For example, when asked about negative aspects of our interface many users choose to respond with ``no'' or ``none''.
However, some participants, whom readily provide feedback, may decide to omit feedback for specific questions to communicate an implied ``no''.
Ambiguous omissions of feedback proses a problem for analysis. 
For example I cannot assume that users imply there are no negative elements on our sketching interface based on user omission of specific feedback questions, although participants may intentionally omitted feedback.
Improvements in how I collect participant feedback need be made to remove ambiguity in omitting feedback.
On a similar note, participants' feedback sometimes does not directly answer corresponding questions asks.
Occasionally, participants' feedback would be more appropriate as the response to another question.
I suspect that participants do no revise feedback after submitting and as a result some of our responses seem similar for multiple questions. \\

Despite all of this, the sketching interface garnered overall positive feedback from our participants.
Many participants, especially the non-experts, claimed that the interface was easy to use and interesting.
 
\section{Model Based Feedback}

\begin{figure}[p]
	\centering
	\includegraphics[width=0.8\textwidth]{model_hist}
	\caption{The distribution of models created on OASIS}
	\label{fig:model_hist}
\end{figure}


\begin{figure}[p]
	\centering
	\includegraphics[width=0.8\textwidth]{renvo_hist}
	\caption{The distribution of renovations created on OASIS}
	\label{fig:renvo_hist}
\end{figure}

There are currently 73 models on OASIS and on average each user generates 1.25 models.
The distribution of the number of models made per user is illustrated in Figure-\ref{fig:model_hist}.
From Figure-\ref{fig:model_hist} we can see that most of our participant only crated a single model.
A handful of participants, however, created more than 9 models on our interface.
While the number of models per users is relatively low, the number of renovations per models show that on average there are 1.9 renovations per model created.
Meaning that about half of our participants renovate their models at least more than once.
Figure-\ref{fig:renvo_hist} show the distribution of models and the number of renovations on % these models.
Analysis of individual models is out of the scope of this thesis, however, I do display some user created models in Figure-\ref{fig:examples}. \\

\begin{figure}[p]
	\centering
	\includegraphics[width=0.5\textwidth]{place_holder}
	\caption{Examples of some users created models on OASIS}
	\label{fig:examples}
\end{figure}

\begin{figure}[p]
	\centering
	\includegraphics[width=0.8\textwidth]{matching_chart}
	\caption{Accuracy in relation to model complexity}
	\label{fig:matching_chart}
\end{figure}

After creating a 3D model participants can voluntarily provide feedback concerning the accuracy of our interpretation.
Participants can state the interpretation of their sketches were either a initially matched their intentions, a match after a performing adjustments as renovations, or not a match at all.
I hypothesized that as models grew more complex the accuracy of our interpretation algorithm would decrease.
Figure-\ref{fig:matching_chart} illustrates model complexity in relation to matching user intentions.
Model complexity is the sum of the number of primitives in a sketch.
Figure-\ref{fig:matching_chart} is interesting because models regardless of complexity seem to always match users initial intentions without requiring renovations.
Even models with 20 to 29 primitives seem to always match.
While the data is strongly indicative that our physical sketch interpretation algorithm is accurate, I believe that more feedback is required before any statically significant conclusions can be drawn.
From Figure-\ref{fig:matching_chart} and the fact that there are 73 models on OASIS, but only fewer responses makes it is clear that participants do not answer this specific feedback question readily.
In order to better gauge our accuracy, changes to the interface need to be made to persuade participants to answer this feedback question. \\

% Aside from categorical quantitative feedback, we also ask users to % quantitatively describe their overall impressions of the system effectiveness % in construction 3D models from user sketches; 
% Table-\ref{} displays results from this feedback questions.
% Of the 16 users that provided feedback on the effectiveness of the physical % sketch interpretation algorithm, 14 stated that the system matched their % intentions.
% One user stated that they only saw 2D versions of the interpretation.
% This could be caused from the user either not rotating their model or limited % WebGL support on their web browser.
% I do  not collect meta data on user web browsers so there currently no % explanation of this.
% Two users mentioned the absence of support for doors hindered the % effectiveness of generating 3D models.
% One user linked us an image in their feedback, that displayed a rendering % issue.
% The detail of some of the qualitative feedback provided from users was much % higher than originally expected.
% Equally important, we asked users to describe cases where models were % incorrectly interpreted by our physical sketching algorithm.
% 
% Table-\ref{} presents the feedback given about these failure cases.
% Eight users provided feedback on failure cases, however, some feedback % provided prove hard to analyze.
% Three users when providing feedback make a reference to problems on specific % models.
% Only one user provided a link to an image of a model they had issue with.
% I suspect the other two users thought OASIS kept track of which model users % were currently viewing when providing feedback.
% This particlar feedback question is a general system wide question, so we did % not anticicapte that users would accociate this question with the model % currently being viewed.
% Additionally, two other responces state there were no issues and another user % misintrpted the question and stated that the lack of lofted beds in the system % was a limiation.
% Another users stated that the transpernt walls only showed windows when % viewing from certain angles.
% The user also stated that they noticed the floor of models was no visiable % when viewing a model from underneath the floor.
% That one specfic concern however, was a design choice I made.
% I belived that if uses wanted to see into their model from the floor upwards % they could only do if the backside of the floor was rendered as non-visable.
% Overall, this question showed me that users require some way to link a failure % case rather then  simply describe it.
% Descriptions of a failure without a model that failed proved to be ambigious % and was unhelpful in  diognosing users' issue.
% 
% \section{Daylighting Analysis Feedback}
% The final set of qualative feedback I collect concerns users experence with % the daylighting portion of our interface.
% Speifically we ask users if they understood the results of daylight % simulations and to desrvibe anything unclear or confusing.
% Table-\ref{} contains user feedback about users' understand of simulation % results.
% Note that the feed back collected in Table-\ref{} was accociated per rendering.
% I anticipcated that users would understand some simulations results for a % model but not undersnad others.
% Surpisingly usrers answers did not very from one rendering to another.
% Out of then user responces 9 responces claimed to understand the simulation % results.
% One user expected daylight from a north facing window at noon in the norther % hemisphere;
% There is direct sunlight from northern fenestrtions in the northern hemisphere % and the user obviouslly know this.
% Users misconceptions about daylight are not in the scope of this thesis, but, % covered in a previous users study on the tangible users interface on the % virtual heliodion\cite{}.
% Lastly we asked users if they system allowed them to test daylighting % perfoamcnce and if they understood over and under illumination visualizations.
% Table-\ref{} shows that of 10 users that provided feedback seven were posative % statements.
% In general users understood over and under illumination and claimed that OASIS % was useful for daylighting preformance and analysis.
% 
% This feedback question is user-specific,however, one user provided feedback % that seemed to be associated with a specific rendering.
% More intrestingly, our most experenced user claimed that they did not see % OASIS was not effective for daylight performance.
% The user did not understand what under illumiiation was relative to.
% In OASIS we do not currentlu support adjusting therehsholds for under and over % illumination.
% Adjusting these thresholds for common activies, such as office work, are left % as future work.
% 






































