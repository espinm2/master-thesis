%%%%%%%%%%%%%%%%%%%%%%%%%%%%%%%%%%%%%%%%%%%%%%%%%%%%%%%%%%%%%%%%%%%
%                                                                 %
%                            CHAPTER TWO                          %
%                                                                 %
%%%%%%%%%%%%%%%%%%%%%%%%%%%%%%%%%%%%%%%%%%%%%%%%%%%%%%%%%%%%%%%%%%%
\chapter{RELATED WORKS} \label{sec:introduction}
% Introduction into the 3 topics below we are about to cover
	% Sentence Hook
	% Sentence 1: Virtual Heliodon
	% Sentence 2: Daylighting Software
	% Sentence 3: Crowd Sourcing User Study
	% Sentence Recap

\section{Virtual Heliodon}
    % Introduction explaining that my work is a extensions to this work
    % SAR sentence
    % Sketch Interface sentence
    % Recap

	\subsection{Daylighting Design with A Spacial Augmented Reality}
		% Introductory Sentence ( Don't know what to say exactly)
		% The augmented reality with projectors
		% LSVO + Daylighting
		% Recap why 

		\paragraph{Augmented Reality with Projectors}
			% mention projectors and coverage
				% slight mention of problems with coverage
			% collaborative space
			% people tokens
			% immerse space + engaging

		\paragraph{LSVO + Daylighting Simulation}
			% mention to radiance
			% mention to why we choose lsvo > radiance
				% speed
				% good results given a small time window
			% mention to how this works with photon mapping and optix
				% mention how it is a gpu thing

	\subsection{Floor Plan Design with A Physical Sketching Interface}
		% Introductory Sentence ( Don't know what to say exactly)
		% 3D modeling types 
		% How we are using sketching borrowed from virtual Heliodon

		\paragraph{3D Modeling}
			% Talk about what parametric modeling is
				% mention HEED
				% mention eQuest
			% Talk about what geometric modeling is
				% Mention sketchup
				% Mention autodesk

			% Talk about sketching modeling
			% Talk about how Barb did this shortly
			% Mention how we use this

		\paragraph{Related Sketching Interfaces}
			% Similar to lightsketch
			% Similar to VR sketchpad proj
			% Similar to erics ref paper

\section{Daylighting Software}
	% Compare Velux
	% Compare Project Versai
	% Compare LightSketch
	% Compare design builder
	% Compare plug-ins
		% Compare ecotect
		% Compare ladybug
	% Wrap things up in this sentence

\section{Crowd sourcing User Studies}
	% Mention crowd sourcing a useful research tool for validations of software.

	% Papers to reference: Crowd sourcing User Studies w/ Mechanical Turk
		% Economics of user studies
			% Time constants
			% low-cost & timely
			% collecting input from only a small set of participants is programmatic in many design situations(even large one) are easily caught with small number of participants

	% Web Credibility Research: A Method for Online Experiments and Early Study Results
		% TODO Information:
			% things you can do to make a website seem credible
			% reduce ads and increase attainability
			% this study shows that online studies are much faster, and much larger then normal studies
			% you can a global demographic

	% Testing Web Sites: Five Users Is Nowhere Near Enough 
		% TODO information:
		% 5 users catch 35% of the problems for websites ( was assumed to 85%)
		% rethinking of our usability engineering testing
		% if we really want to make a useful tools feedback frequently is important

	% Crowd-Sourced Peer Feedback (CPF) for Learning Community Engagement: Results and Reflections from a Pilot Study
		% TODO information:
		% CPF increase engagement, motivation, and learning
		% we might want to use this to teach daylighting
		% we get engaged,motivated, and learn from feedback, so it would good for researchers.

	% Papers to ref: Cookies of Cobblers?
		% Crowds can be leverages to partake in a creative iterative design processes.
		% Maybe we can get crowd sourced arch sketches
		% Enhanced creativity

	% Papers to ref: Real-time Drawing Assistance through Crowd sourcing
		% This paper uses data collected from users to find out what are the important lines drawn on face 
		% closing the loop, the game itself serves as a platform for large-scale evaluation of the effectiveness of our stroke correction algorithm.


%%%%%%%%%%%%%%%%%%%%%%%%%%%%%%%%%%%%%%%%%%%%%%%%%%%%%%%%%%%%%%%%%%%
%                                                                 %
%                            SCRAP YARD                           %
%                                                                 %
%%%%%%%%%%%%%%%%%%%%%%%%%%%%%%%%%%%%%%%%%%%%%%%%%%%%%%%%%%%%%%%%%%%
	% The need for daylighting analysis earlier in the architectural design process, drives the development of tools that generate timely results at affordable levels of effort.
	% In order to perform daylighting analysis, these tools must either import or generate 3D models of architectural spaces.
	% Some tools come packaged with their own parametric or geometric modeling capabilities, however, other tools are available as plug-ins for existing geometric modeling software.
	% Moreover, there are handful of daylighting analysis tools that use sketching inspired interfaces for generating 3D models.
	% All of the analysis tools mentioned below face similar challenges.
	% These tools attempt to provide a means to perform daylighting analysis earlier in the architectural design process.

	% \section{Early Design Phase Parametric-Modeling Daylighting Tools}

	% 	Some early design tools such as HEED and eQUEST allow users to defined architectural spaces with parametric-modeling interfaces. 
	% 	Parametric modeling, as seen in HEED and eQUEST, is the creation of architectural models by defining a set finite parameters.
	% 	These parameters, such as room dimensions, define an architectural space.
	% 	In order to manage the large number of parameters required to analyze energy consumption of a space, both HEED and eQuest use wizards to guide users along the complex processes of defining an architectural space.

	% 	Remarkably, for eQUEST 41 pages of parameters must be defined before conducting analysis on a space given a full 3D model.
	% 	However, parametric modeling doesn't offer much flexibility in design of models.
	% 	HEED and eQUEST are both early design tools that provide users with energy design analytical tools -- including daylighting.
	% 	But because the tools cannot create 
	% 	These tools provide only quantitative data to architects, and do not produce rendering.

	% 	%\subsection{HEED}

	% 	%\subsection{eQUEST}

	% \section{Early Desing Phase Geometric-Modeling Daylighting Tools}
	% 	%\subsection{Sketchup with Lightsolve}
	% 	%\subsection{Autodesk with Ecotect}
	% 	%\subsection{Rhino with Ladybug}
	% 	%\subsection{Velux}
	% 	%\subsection{Daylight-1-2-3}

	% \section{Early Design Phase Sketch-Modeling Daylighting Tools}
	% 	%\subsection{Light Sketch}
	% 	%\subsection{Virtual Heliodon}
