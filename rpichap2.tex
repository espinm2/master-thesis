%%%%%%%%%%%%%%%%%%%%%%%%%%%%%%%%%%%%%%%%%%%%%%%%%%%%%%%%%%%%%%%%%%%
%                                                                 %
%                            CHAPTER TWO                          %
%                                                                 %
%%%%%%%%%%%%%%%%%%%%%%%%%%%%%%%%%%%%%%%%%%%%%%%%%%%%%%%%%%%%%%%%%%%
\chapter{RELATED WORKS} \label{sec:introduction}
% Remeber, this is what we have: 
% 	Sketching, Early Design, Daylighting, Web, User Study
The need for daylighting analysis earlier in the architectural design process, drives the development of tools that generate timely results at affordable levels of effort.
In order to perform daylighting analysis, these tools must either import or generate 3D models of architectural spaces.
Some tools come packaged with their own parametric or geometric modeling capabilities, however, other tools are available as plug-ins for existing geometric modeling software.
Moreover, there are handful of daylighting analysis tools that use sketching inspired interfaces for generating 3D models.
All of the analysis tools mentioned below face similar challenges.
These tools attempt to provide a means to perform daylighting analysis earlier in the architectural design process.


\section{Early Design Phase Parametric-Modeling Daylighting Tools}
	


	%\subsection{HEED}

	%\subsection{eQUEST}

\section{Early Desing Phase Geometric-Modeling Daylighting Tools}
	%\subsection{Sketchup with Lightsolve}
	%\subsection{Autodesk with Ecotect}
	%\subsection{Rhino with Ladybug}
	%\subsection{Velux}
	%\subsection{Daylight-1-2-3}

\section{Early Design Phase Sketch-Modeling Daylighting Tools}
	%\subsection{Light Sketch}
	%\subsection{Virtual Heliodon}
