%%%%%%%%%%%%%%%%%%%%%%%%%%%%%%%%%%%%%%%%%%%%%%%%%%%%%%%%%%%%%%%%%%%
%                                                                 %
%                            CHAPTER ONE                          %
%                                                                 %
%%%%%%%%%%%%%%%%%%%%%%%%%%%%%%%%%%%%%%%%%%%%%%%%%%%%%%%%%%%%%%%%%%%
\chapter{INTRODUCTION} \label{sec:introduction}

\section{Architectural Daylighting}
% TODO

\section{Vitamin D}

Vitamin D is an essential fat-soluble secosteroid required for healthy human functions. It aids in the absorption of calcium and other minerals [IOM].  Vitamin D plays a significant role in the mineralization of bone [IOM].   

Prolonged vitamin D deficiency can result in many serious diseases.
 Adults suffering from vitamin D deficiency can develop osteomalacia -- the softening of bones. Children deprived of vitamin D can develop harmful diseases such as rickets. Children diagnosed with Rickets suffer from poor bone mineralization and are prone to bone fractures and deformity.[Pettifor] 

There are many may to meet daily vitamin D requirements. For example, skin tissue is capable of creating vitamin D on its own, certain foods contain high concentrations of the vitamin, and dietary supplements fortified with vitamin D are readily available.[IOM]

Human skin has a built-in mechanism that helps synthesis vitamin D through the exposure of Ultra Violet(UV) light. Light rich in UV hitting the surface of the skin will begin the processes of vitamin D synthesis. Synthesis through the exposure to sunlight meets most daily vitamin D requirements. Foods we consume are usually rich in vitamin and minerals. However, vitamin D occurs in significant concentrations in very few natural food items, such as fatty fish, particular species of mushrooms, and beef liver.  Because of vitamin D's scarcity in naturally occurring food items and the harmful effects of deficiency vitamin D  in children, companies fortify common breakfast food with vitamin D -- such as orange juice, milk, and cereals. Lastly, Vitamin D can also be taken in pill form as a deity supplement.

Working typical office hours in windowless environments decrease exposure to daylight and increases the risk of vitamin D deficiency. Living an indoors lifestyle coupled with the widespread usage of sunscreen products created a vitamin D deficiency pandemic.  Our skin does not synthesize vitamin D efficiently. Wearing sunscreen with an SPF of 15 absorbs 99\% of UVB radiation and consequently, reduce the ability to synthesize vitamin D by as much as 99\% [Holick].

Architectural daylighting can help alleviate this risk by creating buildings with apertures and geometry that promote deep penetration of natural lighting into a building's interior. Daylight is rich in UV radiation required for vitamin D synthesis.  daylighting systems could, in theory, help occupants keep occupants healthy by passively enabling occupants to meet their daily vitamin D requirements.

\section{Circadian Photobiology}
Daylighting has influence over our circadian photobiology. Circadian photobiology is the human experience hormonal and behavioral changes throughout a roughly 24-hour cycle. The hypothalamic suprachiasmatic nucleus (SCN)  in the brain, which relies on input from non-rod/non-cone photoreceptor systems located in our retina, regulates these non-image forming light responses.[Rea,Thapan]. These non-rod/non-cone photoreceptors are excited by exposure to alternating periods of light and dark. They specifically respond to lighting conditions found in daylight. [Rea, Thapan]

Electrical lighting varies from daylight a couple of biologically important ways [Rea]. Daylight offers a higher levels of illumination, a wider spectrum of electromagnetic radiation, and a temporal variation in lighting. [Robbins]Firstly, sunlight in conjunction with skylight, measures anywhere between 10 to 100 thousand lux. [Robbins]However, the government agency of Occupational Safety and Health Administration (OSHA) set 322 lux as the minimum of lighting requirement for typical office work. [OSHA]Lighting conditions that do not excite photoreceptors responsible for maintaining our circadian rhythm are essentially biological darkness [Leslie].  Secondly, the spectrum of light emitted by artificial lighting lacks short wavelength electromagnetic radiation found in sunlight.Varying wavelengths of electromagnetic radiation affects melatonin levels in humans as much as varying intensity of light.Melatonin suppression is necessary because it plays a role in sleep-wake cycles, body temperature regulation, alertness, and blood pressure.[Joshua J. Gooley] Studies show melatonin suppression varies most through exposure to short wave electromagnetic radiation [Brainard]. Consequently, daylighting systems offer the advantage of exposure to short wavelength electromagnetic radiation needed for melatonin suppression. Lastly, exposure to light during periods of the day asynchronous to our circadian rhythm can result in shifts in our sleep-wake cycles. These shifts, known as phase shifts, triggers melatonin suppression at particular times. For instance,  morning light exposure triggers melatonin suppression resulting in the feeling of alertness. [Rea]However, exposure to light at asynchronous times of day results in a phase shift. An unexpected phase shift can have symptoms similar to jet lag and significantly hinder productivity [Rea]. Daylight availability during those crucial morning hours could potentially have significant impacts on employee productivity. 

\section{Increased Productivity}
Studies show daylighting systems increase both the productivity and comfort of occupants [Menzies].  Daylighting increases workplace productivity and satisfaction through a variety of means. To begin, the human eye as image processing system has evolved over millions of years to work optimally under full spectrum illumination provided by sunlight and skylight. It is not surprising that the human visual system works better using daylight as a source of light[Robbins]. A visual task, such as reading, generally require less illumination when using daylight as opposed to electrical lighting. [Erenkrantz Group 1979] Additionally, daylight provides superior color rendering. Our visual system is tuned to differentiate colors under full spectrum illumination. Differentiating colors under low illumination or fluorescent lighting is not as reliable as compared to daylight. [Robbins] There are current electrical lighting systems that provide full spectrum light, however, these systems are very costly when compared to daylight. Moreover, occupants enjoy being near windows since it gives them information about their outdoor environment -- including the time of day, weather conditions outdoors, and activities happening outside [Leslie].  Having a workstation near a window could evoke a feeling of importance in occupants. This feeling of importance increases worker satisfaction and could possible increase productivity. [Leslie] Overall, the satisfaction of occupants is important to architects and managers, because adverse environmental factors hinder productivity in a workspace. 

These gains provide a financial benefits to companies investing in daylighting systems. However, focus groups and interviews with professionals conducted show that architects prioritize the comfort and productivity of a building’s inhabitants over a building’s sustainability [Menzies]. Meaning building designers see daylighting as means to make occupants comfortable through use of natural lighting, rather than as a eco-friendly lighting system. 

\section{Reduced Energy Demands}
There are direct economic gains from daylighting systems. Energy saving from reducing electrical illumination use save building owners money.
It is important to note that daylighting systems do not directly save capital, rather daylighting systems give building owners the opportunity to conserve energy by using sunlight as an alternative or supplement to electric illumination. Electricity companies charge peak hour rates during the afternoon when demand for electricity is highest. During these hours alternatives sources of light, such as daylighting, become cost effective.
It is hard to estimate how much energy savings with daylighting systems. Simulations are an important tool architects use to determine energy cost saving during the design development processes.  Lighting usually accounts for about 25-40\% of a total building energy demands.
According to one study daylight can save up to 52\% of energy on a wall adjacent to a window [Leslie].

Using daylight as an alternative or supplement to electrical lighting requires some form of daylight management. Daylighting management requires dimming systems that dim electrical lighting during the peak hours when daylight is most available.  Some simulation results show that when there no lighting management in place,  power consumption from lighting can exceed 50\% of a building's total power demand.
However, those simulations also show daylighting can save a building up to 18\% to 55% of a building heating and lighting demand [Bodart].
Without a dimming system, the window of time in which daylighting is cost effective is significantly smaller. Other simulation results showed energy savings of 60\% with daylighting and dimming control strategies [Ihm].

Also, dimming lights result in reduced thermal output from lighting fixtures. Which in turn reduces the total cooling load required in space. The reduced cooling load also contributes to energy saving in daylighting systems [Leslie].  In addition to reducing the cooling load, daylighting can also be used for heat gains during the winter. Daylighting systems exploit the shallow sun angle in the winter months and allow winter sunlight into a building. Heating a large space is expensive, and sunlight can aid in heating [Bodart].

\section{Challenges of Daylighting}

\section{Daylighting strategies and techniques}

