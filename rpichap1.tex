%%%%%%%%%%%%%%%%%%%%%%%%%%%%%%%%%%%%%%%%%%%%%%%%%%%%%%%%%%%%%%%%%%%
%                                                                 %
%                            CHAPTER ONE                          %
%                                                                 %
%%%%%%%%%%%%%%%%%%%%%%%%%%%%%%%%%%%%%%%%%%%%%%%%%%%%%%%%%%%%%%%%%%%

\chapter{INTRODUCTION} \label{sec:introduction}

Daylighting is the use of natural light and building geometry for aesthetically pleasing visuals and the creation of productive environments.
However, daylighting is much more than just pleasing visuals and productive environments.
Daylighting is also an environmental sustainability design practice for the creation of greener buildings and reduction power consumption.
Similarly, daylighting can also be seen an  economic means to reduce a building energy demands or increase worker productivity to generate capital.
Despite the variety of definitions, daylighting will always refer to the use of daylight to met an architectural purpose.

% Overview 
Firstly, to understand what drives daylighting research a brief overview of daylight's advantages is necessary.
In short, daylight is mainly valued as a source of illumination however recent studies show that daylight also offers economic and health benefits.
%
Secondly, I explain why architects struggle with the design of daylighting systems.
By and large, daylighting is challenging by virtue sunlight's dynamic nature.
Moreover, daylight used incorrectly can cause occupants both visual and thermal discomfort.
%
Lastly, I review architectural practices used in the design of daylighting system for the purpose of better following the advances .
Briefly, architects exercise sketching techniques, follow rules-of-thumb, and consult daylighting visualizations to help guide the design of effective daylighting systems. 
%
All things considered, the motives that drive architects and building owners to employ daylighting systems also drive researchers to developer better tools for the design and analysis of daylight in architectural spaces.\\

% This section is basically done
\section{Benefits And Motivations Of Daylighting Systems}
    
  % Our small intro into the next few pages of motives
  There are many benefits to using daylight over traditional electrical lighting.
  Recent studies show exposure to sunlight, offered readily through daylighting systems, has a variety of health benefits; benefits such as the stimulation of vitamin D production and maintenance of healthy circadian rhythms.
  In addition to health-related benefits there are economic motives that drive architects and building owners to to implementing daylighting systems.
  Some economic motives include increases in worker productivity and overall reduced building energy demands.
  In short, daylighting system offer both economic incentives for building owners and health benefits for occupants.

  % TODO: Requires proof reading
  \subsection{Vitamin D}
    Vitamin D is an essential fat-soluble secosteroid required for healthy human functions. It aids in the absorption of calcium and other minerals. Vitamin D plays a significant role in the mineralization of bone\cite{Ross}.   
    Prolonged vitamin D deficiency can result in many serious diseases.
    Adults suffering from vitamin D deficiency can develop osteomalacia -- the softening of bones. Children deprived of vitamin D can develop harmful diseases such as rickets. Children diagnosed with Rickets suffer from poor bone mineralization and are prone to bone fractures and deformity\cite{Pettifor}. \\

    There are many may to meet daily vitamin D requirements. For example, skin tissue is capable of creating vitamin D on its own, certain foods contain high concentrations of the vitamin, and dietary supplements fortified with vitamin D are readily available\cite{Ross}.
    Human skin has a built-in mechanism that helps synthesis vitamin D through the exposure of Ultra Violet(UV) light. Light rich in UV hitting the surface of the skin will begin the processes of vitamin D synthesis. Synthesis through the exposure to sunlight meets most daily vitamin D requirements. Foods we consume are usually rich in vitamin and minerals. However, vitamin D occurs in significant concentrations in very few natural food items, such as fatty fish, particular species of mushrooms, and beef liver.  Because of vitamin D's scarcity in naturally occurring food items and the harmful effects of deficiency vitamin D  in children, companies fortify common breakfast food with vitamin D -- such as orange juice, milk, and cereals. Lastly, Vitamin D can also be taken in pill form as a deity supplement. \\

    Working typical office hours in windowless environments decrease exposure to daylight and increases the risk of vitamin D deficiency. Living an indoors lifestyle coupled with the widespread usage of sunscreen products created a vitamin D deficiency pandemic.  Our skin does not synthesize vitamin D efficiently. Wearing sunscreen with an SPF of 15 absorbs 99\% of UVB radiation and consequently, reduce the ability to synthesize vitamin D by as much as 99\%\cite{Holick}. \\

    Architectural daylighting can help alleviate this risk by creating buildings with apertures and geometry that promote deep penetration of natural lighting into a building's interior. Daylight is rich in UV radiation required for vitamin D synthesis.  Daylighting systems could, in theory, help occupants keep occupants healthy by passively enabling occupants to meet their daily vitamin D requirements. \\

  % TODO: Requires proof reading
  \subsection{Circadian Photobiology}
    Daylighting has influence over our circadian photobiology. Circadian photobiology is the human experience hormonal and behavioral changes throughout a roughly 24-hour cycle. The hypothalamic suprachiasmatic nucleus (SCN)  in the brain, which relies on input from non-rod/non-cone photoreceptor systems located in our retina, regulates these non-image forming light responses. These non-rod/non-cone photoreceptors are excited by exposure to alternating periods of light and dark. They specifically respond to lighting conditions found in daylight\cite{Rea,Thapan}.\\

    Electrical lighting varies from daylight a couple of biologically important ways\cite{Rea}. Daylight offers a higher levels of illumination, a wider spectrum of electromagnetic radiation, and a temporal variation in lighting. Firstly, sunlight in conjunction with skylight, measures anywhere between 10 to 100 thousand lux\cite{Robbins}. However, the government agency of Occupational Safety and Health Administration (OSHA) set 322 lux as the minimum of lighting requirement for typical office work\cite{OSHA}. Lighting conditions that do not excite photoreceptors responsible for maintaining our circadian rhythm are essentially biological darkness\cite{Leslie}. Secondly, the spectrum of light emitted by artificial lighting lacks short wavelength electromagnetic radiation found in sunlight.Varying wavelengths of electromagnetic radiation affects melatonin levels in humans as much as varying intensity of light. Melatonin suppression is necessary because it plays a role in sleep-wake cycles, body temperature regulation, alertness, and blood pressure\cite{Gooley}. Studies show melatonin suppression varies most through exposure to short wave electromagnetic radiation \cite{Brainard}. Consequently, daylighting systems offer the advantage of exposure to short wavelength electromagnetic radiation needed for melatonin suppression. Lastly, exposure to light during periods of the day asynchronous to our circadian rhythm can result in shifts in our sleep-wake cycles. These shifts, known as phase shifts, triggers melatonin suppression at particular times. For instance,  morning light exposure triggers melatonin suppression resulting in the feeling of alertness\cite{Rea}. However, exposure to light at asynchronous times of day results in a phase shift. An unexpected phase shift can have symptoms similar to jet lag and significantly hinder productivity\cite{Rea}. Daylight availability during those crucial morning hours could potentially have significant impacts on employee productivity. \\

  % TODO: Requires proof reading
  \subsection{Increased Productivity}
    Studies show daylighting systems increase both the productivity and comfort of occupants\cite{Menzies}.  Daylighting increases workplace productivity and satisfaction through a variety of means. To begin, the human eye as image processing system has evolved over millions of years to work optimally under full spectrum illumination provided by sunlight and skylight. It is not surprising that the human visual system works better using daylight as a source of light. A visual task, such as reading, generally require less illumination when using daylight as opposed to electrical lighting\cite{Robbins}. Additionally, daylight provides superior color rendering. Our visual system is tuned to differentiate colors under full spectrum illumination. Differentiating colors under low illumination or fluorescent lighting is not as reliable as compared to daylight\cite{Robbins}. There are current electrical lighting systems that provide full spectrum light, however, these systems are very costly when compared to daylight. Moreover, occupants enjoy being near windows since it gives them information about their outdoor environment -- including the time of day, weather conditions outdoors, and activities happening outside. Having a workstation near a window could evoke a feeling of importance in occupants. This feeling of importance increases worker satisfaction and could possible increase productivity\cite{Leslie}. Overall, the satisfaction of occupants is important to architects and managers, because adverse environmental factors hinder productivity in a workspace.  \\

    These gains provide a financial benefits to companies investing in daylighting systems. However, focus groups and interviews with professionals conducted show that architects prioritize the comfort and productivity of a building’s inhabitants over a building’s sustainability\cite{Menzies}. Meaning building designers see daylighting as means to make occupants comfortable through use of natural lighting, rather than as a eco-friendly lighting system. \\

  % TODO: Requires proof reading
  \subsection{Reduced Energy Demands}
    There are direct economic gains from daylighting systems. Energy saving from reducing electrical illumination use save building owners money.
    It is important to note that daylighting systems do not directly save capital, rather daylighting systems give building owners the opportunity to conserve energy by using sunlight as an alternative or supplement to electric illumination. Electricity companies charge peak hour rates during the afternoon when demand for electricity is highest. During these hours alternatives sources of light, such as daylighting, become cost effective.
    It is hard to estimate how much energy savings with daylighting systems. Simulations are an important tool architects use to determine energy cost saving during the design development processes.  Lighting usually accounts for about 25-40\% of a total building energy demands.
    According to one study daylight can save up to 52\% of energy on a wall adjacent to a window\cite{Leslie}.\\

    Using daylight as an alternative or supplement to electrical lighting requires some form of daylight management. Daylighting management requires dimming systems that dim electrical lighting during the peak hours when daylight is most available.  Some simulation results show that when there no lighting management in place,  power consumption from lighting can exceed 50\% of a building's total power demand.
    However, those simulations also show daylighting can save a building up to 18\% to 55\% of a building heating and lighting demand\cite{Bodart}.
    Without a dimming system, the window of time in which daylighting is cost effective is significantly smaller. Other simulation results showed energy savings of 60\% with daylighting and dimming control strategies\cite{Ihm}. \\

    Also, dimming lights result in reduced thermal output from lighting fixtures. Which in turn reduces the total cooling load required in space. The reduced cooling load also contributes to energy saving in daylighting systems\cite{Leslie}. In addition to reducing the cooling load, daylighting can also be used for heat gains during the winter. Daylighting systems exploit the shallow sun angle in the winter months and allow winter sunlight into a building. Heating a large space is expensive, and sunlight can aid in heating\cite{Bodart}. \\

% This requires some work
\section{Challenges Of Designing Daylighting Systems}
  
  % Unsure of apostrophe in "outnumbers design' affect"
  Daylight has many benefits over traditional electrical lighting, however,reaping those benefits is not effortless. There are many factors architects have to consider when designing a daylighting system. Choices made during the early stages of design can have extensive impact on the effectiveness of a daylighting system. Likewise, design choices can also result in visual discomforts for occupants and economic loss for building owners. By and large, architects planning daylighting systems are required to analyze numerous designs' affect on daylight. Furthermore, architects have to be cautious of sunlight's dangers to both occupants and building owners.

  \subsection{Factors That Affect Daylighting}

    % [Introduction]
    Illumination of an architectural space via daylight is dependent on numerous factors including building-wide design choices, room-specific choices, and temporal variations.
    These factors make it difficult to access the quality of a design in terms of daylighting.\\

    \paragraph{Building-wide Design Choices} 
    The cardinal orientation of a building is a choice that directly affect how daylight will illuminate architectural spaces. In the northern hemisphere, windows facing the south cardinal direction experience direct daylight throughout the day. On the other hand, north facing windows do not experience this effect. Rather north facing windows experience indirect diffuse illumination from the sky. The opposite is true in the southern hemisphere. In the south, north facing windows experience direct daylight and south facing windows experience diffuse indirect light. Likewise, windows facing east experience morning sunlight and windows facing west experience evening sunlight. Variations in eastward and westward lighting are a result the sun's eastwards to westwards path across the sky\cite{Robbins}. See Figure~\ref{fig:north_south} for an illustration.

    \begin{figure}[h]
      \centering
      \includegraphics[width=0.25\textwidth]{north_south_fig}
      \caption{This illustration shows why windows facing southward in the northern hemisphere experience direct daylight and windows facing northward do not. It also shows the converse, north facing windows in the southern hemisphere experience direct lighting, however those facing southward do not.} 
      \label{fig:north_south}
    \end{figure}

    Aside from building orientation, building elevation can affect daylighting as well. Varying building elevation can change how daylight illuminates an architectural space. For example, a building located well above sea level will experience a slight difference in daylighting compared to a building below sea level. Daylight usually enters a space either perpendicular to a flat window pane or at a downwards angle starting from the Sun and ending at the floor and walls. However, a skyscraper could potentially have daylight enter a space at an upwards angle towards the ceiling due to its increased elevation.\\


    \begin{figure}[h]
      \centering
      \includegraphics[width=0.5\textwidth]{sun_position}
      \caption{Illustration to show elevations and azimuth used to find the sun's position in the sky} 
      \label{fig:sun_position}
    \end{figure}

    \begin{equation} \label{eq:elevation}
    E = sin^{-1}(sin(\delta) sin(\phi) + cos(\delta) cos(\phi) cos(HRA))
    \end{equation}
    \begin{equation} \label{eq:azimuth}
    A = cos^{-1}( \frac{sin(\delta) cos(\phi) - cos(\delta) sin(\phi) cos(HRA)}{cos(E)})
    \end{equation}

    Just as important as building orientation and elevation, where a building is geographically built has direct impact on daylighting.
    Specifically, the path the sun travels across the sky varies with geographic location and time. 
    Equation-\ref{eq:elevation} and equation-\ref{eq:azimuth} are commonly used in daylighting to calculate the sun's position in the sky. 
    The elevation angle, given by Equation-\ref{eq:elevation}, is the angle between the horizon and solar zenith, as illustrated in figure~\ref{fig:sun_position}. 
    $\delta$ in  equation-\ref{eq:elevation} and equation-\ref{eq:azimuth} refers to the solar declination angle. 
    Lastly, $\phi$ is the latitude of interest in both equations and $HRA$ is the hour angle in local solar time.
    The azimuth angle, as shown in figure-\ref{fig:sun_position}, is the angle between the cardinal north direction and the direction the sun projected down towards the horizon. The azimuth can be found once the elevation angle has been found, as show in equation-\ref{eq:azimuth}.
    As shown in both equations, the suns position in the sky is relative to longitude, latitude, and temporal variables.\\

    % [Building design decisions]
    \paragraph{Room-specific Design Choices} 

    Room-specific design choices also have an impact on the daylight.
    The geometry of an interior space directly affects the distribution of daylight in a room.
    Geometries can be designed to diffuse direct lighting for uniform illumination and occupant comfort.
    Similarly, shading devices and material properties of interior objects can affect daylighting.
    Shading devices, such as blinds can not only help diffuse direct lighting but also help redirect lighting up towards the ceiling, where it can be diffusely reflected back down towards occupants.
    Also, a careful selection of both the color and the material of interior items such furniture, walls, and ceiling can affect daylight's distribution in an interior space. 

    \begin{figure}[h]
      \centering
      \includegraphics[width=0.5\textwidth]{geometry_sketch}
      \caption{Left: a common skylight placement on the roof of a building. The angled roof is designed to let daylight diffuse as it reflects on towards the floor. Right: A light shelf that helps redirect daylight up towards the ceiling, where it can be diffused and reflected back down on towards the floor.}
      \label{fig:geometry_sketch}
    \end{figure}

    In addition to material and shading devices, window placement and size directly influence daylighting.
    Larger windows and skylights allow more light to enter a space, however, poses the risk of over-illumination and glare for occupants inside.
    Likewise, the glazing material used to treat windows can also be used to control the amount and distribution of daylight entering a space.
    The glass used in commercial buildings are glazed to block a significant portion of light from entering a space.
    Glazing are used because direct sunlight would cause over-illumination, thermal discomfort, and harm to the occupants situated near windows.
    Special glazing can also be used to help diffuse lighting up towards the ceiling and away from occupants.
    The choices that architects make in room-specific design significantly affect the daylighting.\\

    % [Temporally]
    \paragraph{Temporal Variation} 

    It is obvious that daylight varies from sun raise to sun set.
    Less obviously, daylight also varies throughout the year.
    The Sun's position in the sky is shallower during winter season than in the summer season. 
    Due to this, during the winter months daylight enters a room at a shallower angle allowing light to travel deeper than in the summer months.

    \begin{figure}[h]
    \centering
    \includegraphics[width=0.5\textwidth]{summer_winter}
    \caption{Top: illustration to visualize the difference in light penetration during the winter and summer seasons. Bottom: a common daylighting technique is extending the roof to block light during the summer season, but not during the winter season.}
    \label{fig:summer_winter}
    \end{figure}

    Architects interested in sustainability, exploit this by extending the roof thus allowing daylight to enter during the winter and blocking direct daylight during the summer as shown in figure-\ref{fig:summer_winter}.
    Weather conditions also play an important role in the distribution and intensity of daylight. 
    During clear days, direct sunlight can enter a room and cause over illumination and glare.
    However during cloudy days, sunlight is diffused by clouds resulting in daylight that is more uniform and diffuse.
    Weather conditions also vary by location, for example in upstate New York, cloudy skies are common, however in Florida clear skies are more frequent.
    A Daylighting systems would be more efficient in locations with clearer skies then in locations where clear skies are uncommon.\\

    Overall, daylight varies due to many factors. It varies depending on temporal factors, room-specific design choices, and building-wide decisions. These numerous factors make the distribution of daylight in a architectural space non-trivial to predict. 
    These difficulties pose a real challenge in the designing of effective daylighting systems.

  \subsection{Adverse Daylighting Effects}

    As previously discussed, daylighting systems offer occupants a variety of benefits. However, poorly implemented daylighting systems can result in discomfort to occupants and increases in a building's energy demand.

    \paragraph{Occupant Discomfort}

    % Over and under illumination
    Human vision can be understood and compared to an image processing systems.
    We require strong contrast and ample illumination to be able to clearly view and process symbols.
    The performance of visual task, such as reading, varies depending on the illumination provided and clarity of the font.
    Under-illumination can make reading difficult and reduce worker productivity\cite{boyce}.
    Under illumination can occur in daylighting systems when daylight available is below a threshold to perform a specific visual task.
    The Occupational Safety and Health Administration (OSHA) set mandatory minimums on illuminations for common settings including offices, hallways, and warehouses to name a few.
    Offices for example require a minimum of 322 lux.
    % Extension: Add in how we compute lux
    Similarly, hallways and warehouses have lower minimums set because there is no need to focus on fine details\cite{OSHA}. \\

    Another visual discomfort that can occur from poor daylighting is glare.
    Glare is a reduction of contrast due a disproportionate amount of illumination from glare sources compared to illumination on a visual task.
    % Introduce the glare index
    Glare is hard to account for in the early design stages of architecture because glare is dependent on not only sources of illumination but also on viewpoint.
    Specifically, there are two main forms of glare -- disability glare and discomfort glare.\cite{Robbins}
    Disability glare occurs when a glare source is intense enough that it rendered the viewer momentary blind. 
    This kind of glare commonly occurs when driving at night and cars are passing in the opposite lane. 
    The strong light emitted from headlights would reduce the contrast of the road ahead and might result in momentary blindness.
    %
    Likewise, discomfort glare is similar to disability glare but much less dangerous. 
    Discomfort glare is also caused from bright glare sources, such as the Sun or light reflected from the Sun, that making visual task difficult to perform.
    Unlike disability glare, discomfort glare does not cause momentary blindness.  
    Prolonged exposure to discomfort glare when focusing on a visual task,however, significantly reduces both worker productivity and worker satisfaction\cite{boyce}. 
    Another visual discomfort, common in office environments, includes veiled reflection.
    Veiled reflections are the result of light reflecting off a surface directly into the eyes of the viewer. 
    For example reading an article from a glossy magazine in direct sunlight is challenging because at certain viewpoints the gloss on the page reflects light into your eyes reducing the contrast between both the black and white letters. 
    Veiled reflections, like glare, are difficult to predict because they are viewpoint dependent. \\

    Lastly, occupants sitting near windows can experience thermal discomfort at certain times of day.
    Daylight can be useful in warming up a space during the winter, however can also cause discomfort during the summer.
    Not only does unattained solar heat gain cause occupants discomfort, solar heat gain can also discomforts building owners.

    \paragraph{Economic Loss}
    
    Another possible adverse product of daylighting systems is unintended solar heat gain. Solar heat gain is the increase in temperature inside a space due to daylight. 
    If too many windows are installed in particular location, a room can experience unintended solar gains.
    To counter solar heat gain, cooling system must work at a higher load then usual resulting in increased energy usage.
    Furthermore, windows unless insulated well can result in heat loss during the winter.
    Rooms with many windows might let in a lot of daylight, but might also come at the cost of increased heating cost during the winter months.\\

    Lastly, occupant behavior can result in lose of investment capital for building owners.
    Occupants exposed to the visual discomforts of daylight can choose use window blinds to block daylight out entirely.
    If blinds are lowered then electrical lighting is used in place of daylight.
    The use of electrical lighting, given available daylight, results in reduced energy savings for the building owners.
    Moreover, daylighting systems are expensive to design and implement and as a result the initial cost is generally greater then using traditional electrical lighting.
    If occupants continuously choose electrical lighting over daylight, the break even point of the initial investment in a daylighting system is pushed back further -- essentially costing the building owner capital.
    Architects are then faced with the challenge of not only making visually pleasing lighting conditions, but also avoiding discomforts caused by daylight.

% This section needs to be written
\section{Daylighting In The Architectural Design Processes}

  % Introduction into subsections
  % Architects face the challenges of designing daylighting systems by using a variety of strategies and techniques.


  \paragraph{Schematic Design Phase} 
  Architects break apart the architectural design processes into 5 stages. 
  The earliest design stage is known as the schematic design phase.
  Prior to designing a building or space architects must first consult with clients and review project goals and specifications during this phase.
  Daylighting affects all stages of the architectural design process, and as a result, requires architects consideration early on.
  After understanding project specifications and project goals architects begin designing by sketching out general building forms and mass. These are rough sketches, however these sketches can be studied to analyze the quality of a design. Shapes and forms of building are usually tackled in a creative and iterative fashion. Sketches still remain widely used during this stage. With enough practice conveying 3D visual concepts through rough sketches is quick and efficient.
  % Figure of what these sketches look like
  Similarly, other mediums can be used to study a design during the schematic design phase such as physical scale models.
  Lastly,the relationships between rooms and spaces are also defined during this phase, including the intended purpose of specific spaces.

  \paragraph{Design Development Phase} 

  Firstly, architects use a set of rules and helpful visuals during the first stage of the architecture design process to guide the . 
  Secondly, once the initial form and design of a project is selected there is another set of tools and devices that help designers develop daylighting system.
  By and large, most challenges in the design of daylighting systems have seen the development of strategies and techniques aimed at alleviating the difficulty posed by designing with daylight.\\

  \subsection{Schematic Design Phase}
    % Intro into the three points we are going to make
    % What is the schematic design phase
    Architects interested in sustainability have many strategies to manage the complexity of creating daylighting.
    Firstly, there are numerous rules-of-thumb aimed to guide the conceptual design of a building to make the best use of daylight.
    Secondly, previous experiences play a significant role in decision making when designing daylighting systems. Lastly, architects during the early design stages still rely on brief analysis of hand drawn sketches to predict lighting behavior in a space. In summary, there are many tools and techniques designs can leverage when building a daylighting system.\\

    \paragraph{Rules-of-Thumb}
    Architects use general rules-of-thumb during the earliest stage of the design process. 
    During the schematic design phase, architects develop the general form, shape, and mass of an architectural space.
    Because the design of a space is an iterative process, where alterations are made until all requirements set by the client are set\cite{Suwa}, any techniques or strategies used to guide the design of daylighting systems need be quick and easy-to-use. Rules-of-thumb are used in conjunction with sketches guide the design process.
    Recent work at the Lighting Research Center validated some common rules-of-thumb architects have used in the design of daylighting systems\cite{Leslie}.
    One such rule validated is the elongation buildings on the east-west axis.
    In addition, another rule validated is the placement of windows high up a wall.
    Having windows high up allows for deeper penetration of daylight into a space.
    Similarly, direct sunlight is best diffused by using shading devices or by bouncing off interior surfaces.
    Moreover, moving visual task closer to windows takes full advantage of daylight.
    However, moving workstations closer to windows increases the risk of glare. A common rule-of-thumb to mitigate is move workstations perpendicular to windows.
    Overall, there are plenty of rules-of-thumb involved in the design of early daylight system.\\

    \paragraph{Experience}
    % Paraphrase: Determining the amount of aperture in the very beginning of schematic design is most often based on a designer’s experience or on rules of thumb


    \paragraph{Visualizations on Hand Drawn Sketches}

    Ideas are though up and written in the form of pencil sketches.
    Architects use sketches to facilitate problem solving. 
    After sketching a idea, architects and look back on their sketch and try to find problems and improve upon their initial sketch.
    Sketches are a great medium because with practice, sketching becomes an easy and fast medium to represent 3D geometries\cite{Suwa}.

  \subsection{Design Development Phase}
    % Intro into the two points we are going to make

    \paragraph{Virtual 3D Models}
    \paragraph{Physical Scale 3D Models}

