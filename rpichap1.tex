%%%%%%%%%%%%%%%%%%%%%%%%%%%%%%%%%%%%%%%%%%%%%%%%%%%%%%%%%%%%%%%%%%%
%                                                                 %
%                            CHAPTER ONE                          %
%                                                                 %
%%%%%%%%%%%%%%%%%%%%%%%%%%%%%%%%%%%%%%%%%%%%%%%%%%%%%%%%%%%%%%%%%%%
\chapter{INTRODUCTION} \label{sec:introduction}

\section{Architectural Daylighting}
% TODO

\section{Vitamin D}

Vitamin D is an essential fat-soluble secosteroid required for healthy human functions. It aids in the absorption of calcium and other minerals.  Vitamin D plays a significant role in the mineralization of bone \cite{Ross}.   

Prolonged vitamin D deficiency can result in many serious diseases.
 Adults suffering from vitamin D deficiency can develop osteomalacia -- the softening of bones. Children deprived of vitamin D can develop harmful diseases such as rickets. Children diagnosed with Rickets suffer from poor bone mineralization and are prone to bone fractures and deformity.[Pettifor] 

There are many may to meet daily vitamin D requirements. For example, skin tissue is capable of creating vitamin D on its own, certain foods contain high concentrations of the vitamin, and dietary supplements fortified with vitamin D are readily available.[IOM]

Human skin has a built-in mechanism that helps synthesis vitamin D through the exposure of Ultra Violet(UV) light. Light rich in UV hitting the surface of the skin will begin the processes of vitamin D synthesis. Synthesis through the exposure to sunlight meets most daily vitamin D requirements. Foods we consume are usually rich in vitamin and minerals. However, vitamin D occurs in significant concentrations in very few natural food items, such as fatty fish, particular species of mushrooms, and beef liver.  Because of vitamin D's scarcity in naturally occurring food items and the harmful effects of deficiency vitamin D  in children, companies fortify common breakfast food with vitamin D -- such as orange juice, milk, and cereals. Lastly, Vitamin D can also be taken in pill form as a deity supplement.

Working typical office hours in windowless environments decrease exposure to daylight and increases the risk of vitamin D deficiency. Living an indoors lifestyle coupled with the widespread usage of sunscreen products created a vitamin D deficiency pandemic.  Our skin does not synthesize vitamin D efficiently. Wearing sunscreen with an SPF of 15 absorbs 99\% of UVB radiation and consequently, reduce the ability to synthesize vitamin D by as much as 99\% [Holick].

Architectural daylighting can help alleviate this risk by creating buildings with apertures and geometry that promote deep penetration of natural lighting into a building's interior. Daylight is rich in UV radiation required for vitamin D synthesis.  daylighting systems could, in theory, help occupants keep occupants healthy by passively enabling occupants to meet their daily vitamin D requirements.

\section{Circadian Photobiology}
Daylighting has influence over our circadian photobiology. Circadian photobiology is the human experience hormonal and behavioral changes throughout a roughly 24-hour cycle. The hypothalamic suprachiasmatic nucleus (SCN)  in the brain, which relies on input from non-rod/non-cone photoreceptor systems located in our retina, regulates these non-image forming light responses.[Rea,Thapan]. These non-rod/non-cone photoreceptors are excited by exposure to alternating periods of light and dark. They specifically respond to lighting conditions found in daylight. [Rea, Thapan]

Electrical lighting varies from daylight a couple of biologically important ways [Rea]. Daylight offers a higher levels of illumination, a wider spectrum of electromagnetic radiation, and a temporal variation in lighting. [Robbins]Firstly, sunlight in conjunction with skylight, measures anywhere between 10 to 100 thousand lux. [Robbins]However, the government agency of Occupational Safety and Health Administration (OSHA) set 322 lux as the minimum of lighting requirement for typical office work. [OSHA]Lighting conditions that do not excite photoreceptors responsible for maintaining our circadian rhythm are essentially biological darkness [Leslie].  Secondly, the spectrum of light emitted by artificial lighting lacks short wavelength electromagnetic radiation found in sunlight.Varying wavelengths of electromagnetic radiation affects melatonin levels in humans as much as varying intensity of light.Melatonin suppression is necessary because it plays a role in sleep-wake cycles, body temperature regulation, alertness, and blood pressure.[Joshua J. Gooley] Studies show melatonin suppression varies most through exposure to short wave electromagnetic radiation [Brainard]. Consequently, daylighting systems offer the advantage of exposure to short wavelength electromagnetic radiation needed for melatonin suppression. Lastly, exposure to light during periods of the day asynchronous to our circadian rhythm can result in shifts in our sleep-wake cycles. These shifts, known as phase shifts, triggers melatonin suppression at particular times. For instance,  morning light exposure triggers melatonin suppression resulting in the feeling of alertness. [Rea]However, exposure to light at asynchronous times of day results in a phase shift. An unexpected phase shift can have symptoms similar to jet lag and significantly hinder productivity [Rea]. Daylight availability during those crucial morning hours could potentially have significant impacts on employee productivity. 

\section{Increased Productivity}
Studies show daylighting systems increase both the productivity and comfort of occupants [Menzies].  Daylighting increases workplace productivity and satisfaction through a variety of means. To begin, the human eye as image processing system has evolved over millions of years to work optimally under full spectrum illumination provided by sunlight and skylight. It is not surprising that the human visual system works better using daylight as a source of light[Robbins]. A visual task, such as reading, generally require less illumination when using daylight as opposed to electrical lighting. [Erenkrantz Group 1979] Additionally, daylight provides superior color rendering. Our visual system is tuned to differentiate colors under full spectrum illumination. Differentiating colors under low illumination or fluorescent lighting is not as reliable as compared to daylight. [Robbins] There are current electrical lighting systems that provide full spectrum light, however, these systems are very costly when compared to daylight. Moreover, occupants enjoy being near windows since it gives them information about their outdoor environment -- including the time of day, weather conditions outdoors, and activities happening outside [Leslie].  Having a workstation near a window could evoke a feeling of importance in occupants. This feeling of importance increases worker satisfaction and could possible increase productivity. [Leslie] Overall, the satisfaction of occupants is important to architects and managers, because adverse environmental factors hinder productivity in a workspace. 

These gains provide a financial benefits to companies investing in daylighting systems. However, focus groups and interviews with professionals conducted show that architects prioritize the comfort and productivity of a building’s inhabitants over a building’s sustainability [Menzies]. Meaning building designers see daylighting as means to make occupants comfortable through use of natural lighting, rather than as a eco-friendly lighting system. 

\section{Reduced Energy Demands}
There are direct economic gains from daylighting systems. Energy saving from reducing electrical illumination use save building owners money.
It is important to note that daylighting systems do not directly save capital, rather daylighting systems give building owners the opportunity to conserve energy by using sunlight as an alternative or supplement to electric illumination. Electricity companies charge peak hour rates during the afternoon when demand for electricity is highest. During these hours alternatives sources of light, such as daylighting, become cost effective.
It is hard to estimate how much energy savings with daylighting systems. Simulations are an important tool architects use to determine energy cost saving during the design development processes.  Lighting usually accounts for about 25-40\% of a total building energy demands.
According to one study daylight can save up to 52\% of energy on a wall adjacent to a window [Leslie].

Using daylight as an alternative or supplement to electrical lighting requires some form of daylight management. Daylighting management requires dimming systems that dim electrical lighting during the peak hours when daylight is most available.  Some simulation results show that when there no lighting management in place,  power consumption from lighting can exceed 50\% of a building's total power demand.
However, those simulations also show daylighting can save a building up to 18\% to 55\% of a building heating and lighting demand [Bodart].
Without a dimming system, the window of time in which daylighting is cost effective is significantly smaller. Other simulation results showed energy savings of 60\% with daylighting and dimming control strategies [Ihm].

Also, dimming lights result in reduced thermal output from lighting fixtures. Which in turn reduces the total cooling load required in space. The reduced cooling load also contributes to energy saving in daylighting systems [Leslie].  In addition to reducing the cooling load, daylighting can also be used for heat gains during the winter. Daylighting systems exploit the shallow sun angle in the winter months and allow winter sunlight into a building. Heating a large space is expensive, and sunlight can aid in heating [Bodart].

\section{Challenges of Daylighting}
% [Introduction]
The natural illumination of an architectural space is dependent on a variety of factors. These factors include design choices made in both site planning and building design. Daylighting also varies temporally.

% [Site planning options]
Choices made during site planning have the significant impact on daylighting.
A building's cardinal orientation is a choice made during site planning and can directly affect how daylighting illuminates a space.
In the northern hemisphere windows facing the south experience daylight throughout the day.
On the other hand, north facing windows do not experience this effect.
Rather north facing windows experience indirect diffuse illumination.
The opposite is true in the southern hemisphere.
In the south, north facing windows experience direct daylight and north facing windows experience diffuse indirect lighting.
% Insert figures 1 into this from moleskin
Furthermore, given the sun rises from east to west, windows facing east experience morning sunlight and those facing west experience evening sunlight.
Aside from building orientation, building elevation can affect daylighting as well.
A building located well above sea level will experience a slight difference in daylighting compared to a building below sea level.
Also, just as important as building orientation, where to you place a building can have an impact on daylighting.
The path the sun travels across the sky varies with geographic location and time.
As shown in equation [num], to predict the suns position in the sky requires longitude, latitude, and temporal information. Also, as mentioned previously depending on which hemisphere an architectural space resides lighting from south and north facing fenestrations will vary.
Choices in site planning have a considerable effect on daylighting systems.
% Insert the equation here

% [Building design decisions]
Building design choices also have an impact on the daylighting in a system.
The geometry of an interior space affects the distribution and reflection of daylight.
Geometries can be designed such as to diffuse direct lighting for a more uniform illumination for the occupants inside.
Another design choice that affects daylighting is shading devices and material properties of interior objects.
Shading devices, such as blinds can not only help diffuse direct lighting but also help redirect lighting away from occupants and up towards the ceiling, where it can be diffused as more uniform lighting.
Also, a careful selection of both the color and material of the interior items such furniture, walls and ceiling can affect daylighting.

In addition to material and shading devices, window placement and size influence daylighting.
Larger windows and skylights allow more light to enter a space, however, poses the risk of over-illumination and glare on occupants inside.
Lastly another factor that controls daylighting is the glazing material used to treat windows.
The glass employed in offices windows are glazed to block a large portion of lighting from entering a space.
Glazing are used because direct sunlight would cause over illumination and be harmful the occupants situated near windows.
Glazing can also be used to help diffuse lighting up towards the ceiling and away from occupants.
The choices that architects make in building design significantly affect the daylighting results.

% [Temporally]
Daylighting also varies temporally.
Moreover, daylight not only varies throughout the day but also throughout the year.
The Suns position in the sky is shallower during winter season than in the summer season. Because of this, daylight enters deeper into a space during the winter months as opposed to during the summer month.
% Figure 2 insert here
Architects exploit this by extending the roof allowing light to enter during the winter shade direct daylight during the summer.
Another factor to consider is weather condition variations. Skies vary from clear to cloudy and sky conditions have a direct impact on daylighting. During clear days direct lighting can enter a room cause over illumination and glare, however during cloudy days, the sun's light is diffused by clouds and results in a more uniform and softer illumination.
Weather conditions also vary by location, for example in upstate New York, cloudy skies are common, however in Florida clear skies are more frequent.

Overall, daylight varies due to many factors. It varies depending on temporal factors, design choices, and site plans. These numerous factors make the illumination a space non-trivial to predict. These difficulties pose a challenge in the designing of daylighting systems.


% %%%%%%%%%%%%%%%%%%%%%%%%%%%%%%%%%%%%%%%%%%%%%%%%%%%%%%%%%%%%%%%%%%%%%%%%%%%%%%%%%%%%%%%%%%
As previously discussed, daylighting systems offer occupants a variety of benefits.

However, poorly implemented daylighting systems can result in discomforts. Some discomforts include glare, veiled reflections, and unintended solar heat gain.

Glare is a reduction of contrast due a disproportionate amount of illumination from glare sources compared to illumination on a visual task.

% Introduce the glare index
Glare's hard to account for in the early design stages of architecture. Glare is dependent on not only sources of illumination but also on viewpoint.

% Where the hell did I get these two, I think I got it from Robbin's source
There are two main forms of glare. Disability glare and discomfort glare.
Disability glare occurs when a glare source is intense enough that it rendered the viewer temporary blind. Such as when driving at night and a car drives passed in the opposite lane with their high beams on. The intense light from the high beams would reduce the contrast of everything else in your view making you temporary blind.

Discomfort glare is similar to disability glare but less dangerous. Discomfort glare is also caused from a bright glare source that makes a visual task more difficult to perform. However with discomfort glare you are not rendered blind, just discomforted.

Another form of glare is veiled reflections.
Vailed reflections are a result of light reflecting off a surface directly into the eyes of the viewer. Reading glossy magazine pages with light directly overhead is challenging because the gloss on the page reflects light from sources overhead into your eyes reducing the contrast between letters. This reduction in contrast makes reading magazines difficult and discomforting.
Veiled reflections are also difficult to deal with because they are viewpoint dependent.

Lastly, another product of poor daylighting systems is unintended solar heat gain. Solar heat gain is the increase in temperature inside a space due to daylight's penetration into a room. If too many windows are installed in particular location, a room can result in unintended solar gains, making occupants inside uncomfortable.

The results of poor daylighting systems make implemented good daylighting systems a challenge. To find the glance between enough daylight to work in, and too little daylight to work with is difficult.


\section{Daylighting strategies and techniques}

