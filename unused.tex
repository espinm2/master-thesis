  \subsection{Assessment of Daylighting Quality}

    There are difficulties in the assessment of daylighting quality of an architectural design.
    Firstly, a variety of useful daylighting analysis tools are available; however, many tools require physical or virtual 3D models that are time-consuming to make.
    Secondly, architects use loosely understood rules-of-thumb and prior experience to obtain favorable daylighting during the critical early stages of design.
    Lastly, architects need run energy analysis on many designs and generate energy saving measurements that meet energy requirements set by clients.
    Overall, the assessment of daylighting quality is a vital step in the design process that comes with many unique challenges. \\

    % Firstly, a variety of useful daylighting analysis tools are available; however, many tools require physical or virtual 3D models that are time-consuming to make.
    Firstly, there are two standard methods for performing daylighting analysis: using a 3D virtual model of an architectural space or the physical construction of a scale model.
    Both methods require tools/techniques to analyses daylighting measurements such as daylight factor, glazing ratios, and illumination.
    The bottleneck in these two variants of analysis is the overall time required to generate models for analysis. The creation of models has seen improvements as architects create 3D virtual models. 
    % Find that one statistic about making physical models > 3d models
    Also, there are many software tools available such as SketchUp and Ecotect that provide useful information for initial design choices at interactive speeds.
    However, these tools and related tools, provide only direct illumination. Indirect light, such as diffuse lighting are more computationally expensive and, as a result, are not interactive\cite{Yancy}.
    Architects fall back to either physical models for quick daylighting analysis during early design stages or are forced to push back analysis to later stages of the architecture design processes.\\

    % Secondly, architects use loosely understood rules and thumb and prior experience to obtain favorable daylighting during the critical early stages of design.
    Secondly, Studies show that architects still rely on rules-of-thumb and previous experience to make major daylighting design choices in the schematic design phase\cite{Galasiu}.
    While previous experience may help guide a building design, relying on loosely understood rules of thumb is not completely reliable.
    Consequently, the Lighting Research Center in 2002 evaluated some common rules-of-thumb and contributed some rules of their own\cite{Leslie}.
    Evaluated rules-of-thumb and previous experiences still only help guide the design of new daylighting systems--they provide little to no analysis and cannot be used to access the quality of a space.\\

    % Main challenge is meeting energy demands
    Lastly, the design of daylighting systems is difficult because of the challenge of meeting energy requirements set by clients.
    % Why is it difficult to mean energy demands
    Meeting energy demands requires predicting a building energy cost before construction.
    While there are analytic tools available to estimate energy cost, they need both complex 3D models and significant amounts of time for accurate results.
    One of the simplest and quickest tools to use is Daylight 1-2-3.
    Daylight 1-2-3 is an energy analysis tool for designers interested in daylighting.
    While Daylight 1-2-3 is practical for generating monthly energy savings and includes an easy-to-use graphical interface, the tool still requires the creation of a full 3D model and computation time in the order of minutes. \cite{Reinhart}
    The schematic design phase of the architecture design process is done in an iterative fashion.
    Designers need to be able to generate quick 3D models and run analysis at interactive rates.
    % What is the current state of this
    Moreover, in a 2008 a survey of architects, engineers, and researchers involved with daylighting expressed a demand for simple calculations to determine energy savings from daylighting systems.\cite{Galasiu} 
    % Ending
    Overall meeting energy requirements pose a challenge in the design of daylighting systems. \\


    \paragraph{Rules-of-Thumb}
    Architects use general rules-of-thumb during the earliest stage of the design process. 
    During the schematic design phase, architects develop the general form, shape, and mass of an architectural space.
    Because the design of a space is an iterative process, where alterations are made until all requirements set by the client are set\cite{Suwa}, any techniques or strategies used to guide the design of daylighting systems need be quick and easy-to-use. Rules-of-thumb are used in conjunction with sketches guide the design process.
    Recent work at the Lighting Research Center validated some common rules-of-thumb architects have used in the design of daylighting systems\cite{Leslie}.
    One such rule validated is the elongation buildings on the east-west axis.
    In addition, another rule validated is the placement of windows high up a wall.
    Having windows high up allows for deeper penetration of daylight into a space.
    Similarly, direct sunlight is best diffused by using shading devices or by bouncing off interior surfaces.
    Moreover, moving visual task closer to windows takes full advantage of daylight.
    However, moving workstations closer to windows increases the risk of glare. A common rule-of-thumb to mitigate is move workstations perpendicular to windows.
    Overall, there are plenty of rules-of-thumb involved in the design of early daylight system.\\

    \paragraph{Experience}
    % Paraphrase: Determining the amount of aperture in the very beginning of schematic design is most often based on a designer’s experience or on rules of thumb


    \paragraph{Visualizations on Hand Drawn Sketches}

    Ideas are though up and written in the form of pencil sketches.
    Architects use sketches to facilitate problem solving. 
    After sketching a idea, architects and look back on their sketch and try to find problems and improve upon their initial sketch.
    Sketches are a great medium because with practice, sketching becomes an easy and fast medium to represent 3D geometries\cite{Suwa}.

  \subsection{Design Development Phase}
    % Intro into the two points we are going to make

    \paragraph{Virtual 3D Models}
    \paragraph{Physical Scale 3D Models}



  Firstly, architects use a set of rules and helpful visuals during the first stage of the architecture design process to guide the 
  Secondly, once the initial form and design of a project is selected there is another set of tools and devices that help designers develop daylighting system.
  By and large, most challenges in the design of daylighting systems have seen the development of strategies and techniques aimed at alleviating the difficulty posed by designing with daylight.\\

  \subsection{Schematic Design Phase}
    % Intro into the three points we are going to make
    % What is the schematic design phase
    Architects interested in sustainability have many strategies to manage the complexity of creating daylighting.
    Firstly, there are numerous rules-of-thumb aimed to guide the conceptual design of a building to make the best use of daylight.
    Secondly, previous experiences play a significant role in decision making when designing daylighting systems. Lastly, architects during the early design stages still rely on brief analysis of hand drawn sketches to predict lighting behavior in a space. In summary, there are many tools and techniques designs can leverage when building a daylighting system.\\


  Architects break apart the architectural design processes into 5 stages. 
  The earliest design stage is known as the schematic design phase.
  Prior to designing a building or space architects must first consult with clients and review project goals and specifications during this phase.
  Daylighting affects all stages of the architectural design process, and as a result, requires architects consideration early on.
  After understanding project specifications and project goals architects begin designing by sketching out general building forms and mass. These are rough sketches, however these sketches can be studied to analyze the quality of a design. Shapes and forms of building are usually tackled in a creative and iterative fashion. Sketches still remain widely used during this stage. With enough practice conveying 3D visual concepts through rough sketches is quick and efficient.
  % Figure of what these sketches look like
  Similarly, other mediums can be used to study a design during the schematic design phase such as physical scale models.
  Lastly,the relationships between rooms and spaces are also defined during this phase, including the intended purpose of specific spaces.
  At the end of the schematic design phase the architect reviews designs created with clients before moving on to the design development phase.

  \paragraph{Design Development Phase} 
  The design development phase follows right after the schematic design phase and approval from clients.
  In the Design development phase details are fleshed out from the designs developed in the schematic design phase.
  Architects focus on plumbing, electricity, and heating and cooling systems implementation.
  As well a refinements in architectural details, including the exact placements of windows and doorways.
  Usually in this phase a 3D model is generated in software to show clients and help move one step towards the creation of construction documents.

  \paragraph{Construction Documents, Bidding, Construction Administration} 

  % lays out plumbing,electricity,structural, and arch details
  % specific locations of windows and doors
  % specific material types
  % generation of detailed 3D model


% Introduction into the 3 topics below we are about to cover
  OASIS is an extension of previous research on a spatial augmented tangible user interface for daylighting simulations\cite{barb,josh,yushan,everyone}.
  Furthermore, OASIS shares similar goals with existing architecture software for daylighting analysis.
  OASIS also attempts to emulate previous studies successful in crowd sourcing the process of validation and gathering feedback.

\section{Virtual Heliodon}
The Virtual Heliodon is a spatial augmented reality with a tangible user interface for the early collaborative design of interior spaces with daylighting\cite{barb, josh,yushang}. 
Succinctly the Virtual Heliodon is composed of multiple projectors, a circular table, and a collection of styrofoam primitives.  
A large chassis holds the projectors above the table top at evenly spaced intervals. The configuration of the system is shown below in figure-\ref{}


%     At it's core, OASIS is an alternative interface for the Virtual Heliodon\cite{barb}.
%     The Virtual Heliodon gives users the ability to create rough architectural sketches through the physical manipulation of wall primitives.
%     These physical sketches are turned into 3D geometries via a sketch interpretation algorithm\cite{barb}.
%     Also, the Virtual Heliodon uses an in-house daylighting simulating engine to create renderings at interface speeds\cite{josh,yushan}. The daylighting rendering are then projected into a space to create an collaborative augmented reality for users to evaluate architectural sketches\cite{josh}.
%     Overall, OASIS used both the sketching interpretation algorithm, and the daylighting rendering engine used in the Virtual Heliodon.

%   \subsection{Daylighting Design with A Spacial Augmented Reality}
%     % Introductory Sentence ( Don't know what to say exactly)
%     % The augmented reality with projectors
%     % LSVO + Daylighting
%     % Recap why 

%     \paragraph{Augmented Reality with Projectors}
%       % mention projectors and coverage
%         % slight mention of problems with coverage
%       % collaborative space
%       % people tokens
%       % immerse space + engaging

%     \paragraph{LSVO + Daylighting Simulation}
%       % mention to radiance
%       % mention to why we choose lsvo > radiance
%         % speed
%         % good results given a small time window
%       % mention to how this works with photon mapping and optix
%         % mention how it is a gpu thing

%   \subsection{Floor Plan Design with A Physical Sketching Interface}
%     % Introductory Sentence ( Don't know what to say exactly)
%     % 3D modeling types 
%     % How we are using sketching borrowed from virtual Heliodon

%     \paragraph{3D Modeling}
%       % Talk about what parametric modeling is
%         % mention HEED
%         % mention eQuest
%       % Talk about what geometric modeling is
%         % Mention sketchup
%         % Mention autodesk

%       % Talk about sketching modeling
%       % Talk about how Barb did this shortly
%       % Mention how we use this

%     \paragraph{Related Sketching Interfaces}
%       % Similar to lightsketch
%       % Similar to VR sketchpad proj
%       % Similar to erics ref paper

% \section{Daylighting Software}
%   % Compare Velux
%   % Compare Project Versai
%   % Compare LightSketch
%   % Compare design builder
%   % Compare plug-ins
%     % Compare ecotect
%     % Compare ladybug
%   % Wrap things up in this sentence

% \section{Crowd sourcing User Studies}
  % Mention crowd sourcing a useful research tool for validations of software.

  % Papers to reference: Crowd sourcing User Studies w/ Mechanical Turk
    % Economics of user studies
      % Time constants
      % low-cost & timely
      % collecting input from only a small set of participants is programmatic in many design situations(even large one) are easily caught with small number of participants

  % Web Credibility Research: A Method for Online Experiments and Early Study Results
    % TODO Information:
      % things you can do to make a website seem credible
      % reduce ads and increase attainability
      % this study shows that online studies are much faster, and much larger then normal studies
      % you can a global demographic

  % Testing Web Sites: Five Users Is Nowhere Near Enough 
    % TODO information:
    % 5 users catch 35% of the problems for websites ( was assumed to 85%)
    % rethinking of our usability engineering testing
    % if we really want to make a useful tools feedback frequently is important

  % Crowd-Sourced Peer Feedback (CPF) for Learning Community Engagement: Results and Reflections from a Pilot Study
    % TODO information:
    % CPF increase engagement, motivation, and learning
    % we might want to use this to teach daylighting
    % we get engaged,motivated, and learn from feedback, so it would good for researchers.

  % Papers to ref: Cookies of Cobblers?
    % Crowds can be leverages to partake in a creative iterative design processes.
    % Maybe we can get crowd sourced arch sketches
    % Enhanced creativity

  % Papers to ref: Real-time Drawing Assistance through Crowd sourcing
    % This paper uses data collected from users to find out what are the important lines drawn on face 
    % closing the loop, the game itself serves as a platform for large-scale evaluation of the effectiveness of our stroke correction algorithm.


%%%%%%%%%%%%%%%%%%%%%%%%%%%%%%%%%%%%%%%%%%%%%%%%%%%%%%%%%%%%%%%%%%%
%                                                                 %
%                            SCRAP YARD                           %
%                                                                 %
%%%%%%%%%%%%%%%%%%%%%%%%%%%%%%%%%%%%%%%%%%%%%%%%%%%%%%%%%%%%%%%%%%%
  % The need for daylighting analysis earlier in the architectural design process, drives the development of tools that generate timely results at affordable levels of effort.
  % In order to perform daylighting analysis, these tools must either import or generate 3D models of architectural spaces.
  % Some tools come packaged with their own parametric or geometric modeling capabilities, however, other tools are available as plug-ins for existing geometric modeling software.
  % Moreover, there are handful of daylighting analysis tools that use sketching inspired interfaces for generating 3D models.
  % All of the analysis tools mentioned below face similar challenges.
  % These tools attempt to provide a means to perform daylighting analysis earlier in the architectural design process.

  % \section{Early Design Phase Parametric-Modeling Daylighting Tools}

  %   Some early design tools such as HEED and eQUEST allow users to defined architectural spaces with parametric-modeling interfaces. 
  %   Parametric modeling, as seen in HEED and eQUEST, is the creation of architectural models by defining a set finite parameters.
  %   These parameters, such as room dimensions, define an architectural space.
  %   In order to manage the large number of parameters required to analyze energy consumption of a space, both HEED and eQuest use wizards to guide users along the complex processes of defining an architectural space.

  %   Remarkably, for eQUEST 41 pages of parameters must be defined before conducting analysis on a space given a full 3D model.
  %   However, parametric modeling doesn't offer much flexibility in design of models.
  %   HEED and eQUEST are both early design tools that provide users with energy design analytical tools -- including daylighting.
  %   But because the tools cannot create 
  %   These tools provide only quantitative data to architects, and do not produce rendering.

  %   %\subsection{HEED}

  %   %\subsection{eQUEST}

  % \section{Early Desing Phase Geometric-Modeling Daylighting Tools}
  %   %\subsection{Sketchup with Lightsolve}
  %   %\subsection{Autodesk with Ecotect}
  %   %\subsection{Rhino with Ladybug}
  %   %\subsection{Velux}
  %   %\subsection{Daylight-1-2-3}

  % \section{Early Design Phase Sketch-Modeling Daylighting Tools}
  %   %\subsection{Light Sketch}
  %   %\subsection{Virtual Heliodon}
