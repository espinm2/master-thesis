  \subsection{Assessment of Daylighting Quality}

    There are difficulties in the assessment of daylighting quality of an architectural design.
    Firstly, a variety of useful daylighting analysis tools are available; however, many tools require physical or virtual 3D models that are time-consuming to make.
    Secondly, architects use loosely understood rules-of-thumb and prior experience to obtain favorable daylighting during the critical early stages of design.
    Lastly, architects need run energy analysis on many designs and generate energy saving measurements that meet energy requirements set by clients.
    Overall, the assessment of daylighting quality is a vital step in the design process that comes with many unique challenges. \\

    % Firstly, a variety of useful daylighting analysis tools are available; however, many tools require physical or virtual 3D models that are time-consuming to make.
    Firstly, there are two standard methods for performing daylighting analysis: using a 3D virtual model of an architectural space or the physical construction of a scale model.
    Both methods require tools/techniques to analyses daylighting measurements such as daylight factor, glazing ratios, and illumination.
    The bottleneck in these two variants of analysis is the overall time required to generate models for analysis. The creation of models has seen improvements as architects create 3D virtual models. 
    % Find that one statistic about making physical models > 3d models
    Also, there are many software tools available such as SketchUp and Ecotect that provide useful information for initial design choices at interactive speeds.
    However, these tools and related tools, provide only direct illumination. Indirect light, such as diffuse lighting are more computationally expensive and, as a result, are not interactive\cite{Yancy}.
    Architects fall back to either physical models for quick daylighting analysis during early design stages or are forced to push back analysis to later stages of the architecture design processes.\\

    % Secondly, architects use loosely understood rules and thumb and prior experience to obtain favorable daylighting during the critical early stages of design.
    Secondly, Studies show that architects still rely on rules-of-thumb and previous experience to make major daylighting design choices in the schematic design phase\cite{Galasiu}.
    While previous experience may help guide a building design, relying on loosely understood rules of thumb is not completely reliable.
    Consequently, the Lighting Research Center in 2002 evaluated some common rules-of-thumb and contributed some rules of their own\cite{Leslie}.
    Evaluated rules-of-thumb and previous experiences still only help guide the design of new daylighting systems--they provide little to no analysis and cannot be used to access the quality of a space.\\

    % Main challenge is meeting energy demands
    Lastly, the design of daylighting systems is difficult because of the challenge of meeting energy requirements set by clients.
    % Why is it difficult to mean energy demands
    Meeting energy demands requires predicting a building energy cost before construction.
    While there are analytic tools available to estimate energy cost, they need both complex 3D models and significant amounts of time for accurate results.
    One of the simplest and quickest tools to use is Daylight 1-2-3.
    Daylight 1-2-3 is an energy analysis tool for designers interested in daylighting.
    While Daylight 1-2-3 is practical for generating monthly energy savings and includes an easy-to-use graphical interface, the tool still requires the creation of a full 3D model and computation time in the order of minutes. \cite{Reinhart}
    The schematic design phase of the architecture design process is done in an iterative fashion.
    Designers need to be able to generate quick 3D models and run analysis at interactive rates.
    % What is the current state of this
    Moreover, in a 2008 a survey of architects, engineers, and researchers involved with daylighting expressed a demand for simple calculations to determine energy savings from daylighting systems.\cite{Galasiu} 
    % Ending
    Overall meeting energy requirements pose a challenge in the design of daylighting systems. \\


    \paragraph{Rules-of-Thumb}
    Architects use general rules-of-thumb during the earliest stage of the design process. 
    During the schematic design phase, architects develop the general form, shape, and mass of an architectural space.
    Because the design of a space is an iterative process, where alterations are made until all requirements set by the client are set\cite{Suwa}, any techniques or strategies used to guide the design of daylighting systems need be quick and easy-to-use. Rules-of-thumb are used in conjunction with sketches guide the design process.
    Recent work at the Lighting Research Center validated some common rules-of-thumb architects have used in the design of daylighting systems\cite{Leslie}.
    One such rule validated is the elongation buildings on the east-west axis.
    In addition, another rule validated is the placement of windows high up a wall.
    Having windows high up allows for deeper penetration of daylight into a space.
    Similarly, direct sunlight is best diffused by using shading devices or by bouncing off interior surfaces.
    Moreover, moving visual task closer to windows takes full advantage of daylight.
    However, moving workstations closer to windows increases the risk of glare. A common rule-of-thumb to mitigate is move workstations perpendicular to windows.
    Overall, there are plenty of rules-of-thumb involved in the design of early daylight system.\\

    \paragraph{Experience}
    % Paraphrase: Determining the amount of aperture in the very beginning of schematic design is most often based on a designer’s experience or on rules of thumb


    \paragraph{Visualizations on Hand Drawn Sketches}

    Ideas are though up and written in the form of pencil sketches.
    Architects use sketches to facilitate problem solving. 
    After sketching a idea, architects and look back on their sketch and try to find problems and improve upon their initial sketch.
    Sketches are a great medium because with practice, sketching becomes an easy and fast medium to represent 3D geometries\cite{Suwa}.

  \subsection{Design Development Phase}
    % Intro into the two points we are going to make

    \paragraph{Virtual 3D Models}
    \paragraph{Physical Scale 3D Models}



  Firstly, architects use a set of rules and helpful visuals during the first stage of the architecture design process to guide the 
  Secondly, once the initial form and design of a project is selected there is another set of tools and devices that help designers develop daylighting system.
  By and large, most challenges in the design of daylighting systems have seen the development of strategies and techniques aimed at alleviating the difficulty posed by designing with daylight.\\

  \subsection{Schematic Design Phase}
    % Intro into the three points we are going to make
    % What is the schematic design phase
    Architects interested in sustainability have many strategies to manage the complexity of creating daylighting.
    Firstly, there are numerous rules-of-thumb aimed to guide the conceptual design of a building to make the best use of daylight.
    Secondly, previous experiences play a significant role in decision making when designing daylighting systems. Lastly, architects during the early design stages still rely on brief analysis of hand drawn sketches to predict lighting behavior in a space. In summary, there are many tools and techniques designs can leverage when building a daylighting system.\\


