%%%%%%%%%%%%%%%%%%%%%%%%%%%%%%%%%%%%%%%%%%%%%%%%%%%%%%%%%%%%%%%%%%%
%                                                                 %
%                            ABSTRACT                             %
%                                                                 %
%%%%%%%%%%%%%%%%%%%%%%%%%%%%%%%%%%%%%%%%%%%%%%%%%%%%%%%%%%%%%%%%%%%
\specialhead{ABSTRACT}
Daylighting plays a significant role in architecture; its creative and efficient use offers aesthetics visuals, increased productivity, and energy savings. However, poor implementation of daylighting systems can have adverse impacts such as visual discomfort, solar heat gain, and an absence of energy savings. 

As a result, architects turn to daylighting analysis as means to predict daylighting's effects on architectural spaces prior to construction. However, there are several challenges in daylighting analysis, that make prediction non-trivial and time intensive. Specifically, there are numerous factors to consider when visualizing the natural lighting of an interior space. Daylight can vary depending on the season, the time of day, the cardinal direction of fenestrations,  the geographic location and geometry the space,  the reflectance of interior materials, and more.  

The traditional approaches to solving this problem require either the construction of physical scale model or development of virtual 3D models. Both methods are time intensive and can cause delays in the fast-paced schematic design phase of architecture.  

I present a novel interface that is easily accessible to non-experts providing them with the ability to generate 3D models for daylighting simulation from 2D architectural sketches. This online interface allows users to both quickly create 3D models and analysis daylighting simulation results. I propose that this interface will aid both experts and non-experts during the schematic design phase where ease of expressing 3D geometries and speed of analyzing simulation results is most significant. 

My contributions includes the development of this online interface, the conduction of a large-scale user study, and the analysis of that study.
