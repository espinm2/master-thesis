%%%%%%%%%%%%%%%%%%%%%%%%%%%%%%%%%%%%%%%%%%%%%%%%%%%%%%%%%%%%%%%%%%%
%                                                                 %
%                            ABSTRACT                             %
%                                                                 %
%%%%%%%%%%%%%%%%%%%%%%%%%%%%%%%%%%%%%%%%%%%%%%%%%%%%%%%%%%%%%%%%%%%
\specialhead{ABSTRACT}
Daylighting plays a significant role in architecture.
Daylight's creative and efficient use offers aesthetic visuals, increased productivity, and reduced energy demand. 
However, daylight can also have adverse effects such as visual discomfort, solar heat gain, and an absence of energy savings. \\

As a result, architects turn to daylight analysis to predict daylight's effects on architectural spaces. However, there are several challenges in daylight analysis that make prediction non-trivial and time intensive. Specifically, there are numerous factors to consider when visualizing daylight in an interior space. Daylight can vary depending on the season, the time of day, the cardinal direction of windows, the geographic location , the spatial geometry, and the reflectance of materials. \\

Traditional approaches to daylight analysis require either the construction of physical scale model or development of virtual 3D models. Both methods are time intensive and can cause delays in the fast-paced early design phase of architecture. \\

I present a novel online sketching interface for simulations (OASIS) that is easily accessible to non-experts, providing them with the ability to generate 3D models for daylight simulation from 2D architectural sketches. 
This online sketching interface allows users to both quickly create 3D models and perform qualitative daylight analysis.
I propose that the online sketching interface for simulations(OASIS) is accessible and easy-to-use for both experts and novices.
Additional, I speculate that OASIS a step in the right direction as an early design tool for daylighting analysis.\\

% ASK_BARB: this is wordy, ask her why it is wordy
My contributions include the development of OASIS, the conduction of a pilot user study, and the analysis of results from that study. \\
%%%%%%%%%%%%%%%%%%%%%%%%%%%%%%%%%%%%%%%%%%%%%%%%%%%%%
