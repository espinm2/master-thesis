%%%%%%%%%%%%%%%%%%%%%%%%%%%%%%%%%%%%%%%%%%%%%%%%%%%%%%%%%%%%%%%%%%%
%                                                                 %
%                            ABSTRACT                             %
%                                                                 %
%%%%%%%%%%%%%%%%%%%%%%%%%%%%%%%%%%%%%%%%%%%%%%%%%%%%%%%%%%%%%%%%%%%
\specialhead{ABSTRACT}
Daylighting plays a significant role in architecture; its creative and efficient use offers aesthetics visuals, increased productivity, and reduced energy demand. However, poor implementation of daylighting systems can have adverse impacts such as visual discomfort, solar heat gain, and an absence of energy savings. 

As a result, architects turn to daylighting analysis as means to predict daylighting's effects on architectural spaces prior to construction. However, there are several challenges in daylighting analysis, that make prediction non-trivial and time intensive. Specifically, there are numerous factors to consider when visualizing the natural lighting of an interior space. Daylight can vary depending on the season, the time of day, the cardinal direction of fenestrations,  the geographic location and geometry the space,  the reflectance of interior materials, and more.  

The traditional approaches to solving this problem require either the construction of physical scale model or development of virtual 3D models. Both methods are time intensive and can cause delays in the fast-paced schematic design phase of architecture.  

I present a novel interface that is easily accessible to non-experts providing them with the ability to generate 3D models for daylighting simulation from 2D architectural sketches. This online interface allows users to both quickly create 3D models and analysis daylighting simulation results. I propose that this interface will aid both experts and non-experts during the schematic design phase where ease of expressing 3D geometries and speed of analyzing simulation results is most significant. 

My contributions includes the development of this online interface, the conduction of a large-scale user study, and the analysis of that study.

%%%%%%%%%%%%%%%%%%%%%%%%%%%%%%%%%%%%%%%%%%%%%%%%%%%%%
  Architects break apart the architectural design processes into 5 stages. 
  The earliest design stage is known as the schematic design phase.
  Prior to designing a building or space architects must first consult with clients and review project goals and specifications during this phase.
  Daylighting affects all stages of the architectural design process, and as a result, requires architects consideration early on.
  After understanding project specifications and project goals architects begin designing by sketching out general building forms and mass. These are rough sketches, however these sketches can be studied to analyze the quality of a design. Shapes and forms of building are usually tackled in a creative and iterative fashion. Sketches still remain widely used during this stage. With enough practice conveying 3D visual concepts through rough sketches is quick and efficient.
  % Figure of what these sketches look like
  Similarly, other mediums can be used to study a design during the schematic design phase such as physical scale models.
  Lastly,the relationships between rooms and spaces are also defined during this phase, including the intended purpose of specific spaces.
  At the end of the schematic design phase the architect reviews designs created with clients before moving on to the design development phase.

  \paragraph{Design Development Phase} 
  The design development phase follows right after the schematic design phase and approval from clients.
  In the Design development phase details are fleshed out from the designs developed in the schematic design phase.
  Architects focus on plumbing, electricity, and heating and cooling systems implementation.
  As well a refinements in architectural details, including the exact placements of windows and doorways.
  Usually in this phase a 3D model is generated in software to show clients and help move one step towards the creation of construction documents.

  \paragraph{Construction Documents, Bidding, Construction Administration} 

  % lays out plumbing,electricity,structural, and arch details
  % specific locations of windows and doors
  % specific material types
  % generation of detailed 3D model
