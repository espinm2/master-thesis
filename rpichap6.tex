\chapter{Concuslion and Future Works} \label{sec:results}

We performed a pilot user study to determine if OASIS was accessible, easy-to-use, and a step in the right direction as an early design tool for daylighting.
Feedback from the study that spanned only 2 weeks, demonstrates OASIS was aceesable to a board range of users outside of the traddition pool or achitecture students used in previous user studies.
Moreover, I believe that we have created an accessable tool, as no one in our feedback responces mentioned problems with connecting and using OASIS.
User feedback from our pilot study generally showed that OASIS was easy to use.
However it is also important to note that our user base concist mostly of non-experts. 
Whether our sketching interface was easy to use for experts remains unknown.
Lastly, feedback was collected from users concerning the effectivness of OASIS for daylighting analysis.
Our one users with 10+ years experece stated that OASIS was not useful for daylight analysis.
The user, however, was not clear as to why OASIS was not effective for daylighting analysis.
Before any soild conclusion can be drawn, I believe we need to collect more feedback from experts and advertise our online tool towards users with arichtectural experence.
As it stands I can confidently say we created an accessable tool that is easy to use for non-experts to perform purley qualantive daylighting analysis.

Despite our original hyptothesis not being met,
 out pilot user study did provide us feedback about what usabilty features to prioritize.
Firstly, the most requested feature was doors in our sketching interface.
As mentioned previously, the physical sketch interpration algorithm can interpet multiroom sketches.
However the lack of doors on the sketching interface comminiucated that we do not support multple rooms to many users.
Additionally the study revealed that room designs were greatly limited by the small number of furniture tiems in the system.
Providing user a wider varity of furniture items would aide users in the design of spaces other than bedrooms.
Another common conern was scale, users wanted either control of scale in their sketches or more concretes sense of scale.
Giviing users control of scale and overaly gridlines on the sketching canvcas would better comminucate sizes than our currently indirect sense of scale through statisically sized furinutire items.
Another feature I found surprising was the demand of drag and drop control of walls anfter inital placement.
Before implemementing this feature, however, I would suggest AB testing between our current sketching neviroment and drop and dropping walls.

Aside from futurework on features, out pilot user study provided insight into improvments on future user studies using OASIS.
Firstly, the amount of users who regisited and lost intrrest in our tool is high.
I do not have any data on users rentation rate for online arichtectural tool, however future work can hopefully reduce the number of users who sign up but do not use OASIS.
Such work includes building in objective into using our tool.
Disguising user studies as goal driven games has had sucssus in crowd sourching line drawings in previous research\cite{}.
Letting users use OASIS to fix a problem caused by daylighting with a scoring method might provide an incentive to use OASIS as a daylight analysi tool.
Additional, making reigisitration easier or even optinal could increase user rentation.
And as mentioned previouslyu better targeting of users with arichtecture experence could provide more insightful feedback.
Reddit, while being a large comminity based soical media outlet, it not easy to target ebcause post of tools or research related content generally does not gain much visalibity.
Gameification of OASIS could potentionally incrase user rentatnin.
%include precent ommition ambigiouty
%make it easier to link model
