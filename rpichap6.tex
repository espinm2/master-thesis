\chapter{Conclusion and Future Works} \label{sec:con}

% ===============================================
% we are going to list all of our findings
% Polish...
% ===============================================

	Within the first two weeks of our pilot user study  57 participants registered with our online architectural sketching interface for simulations (OASIS).
	The large number of users who registered with OASIS is encouraging because participants using our tool and providing feedback helps evaluate current features and guide future design on OASIS.
	OASIS is an early architectural design tool; however, OASIS can also be seen as the start of an interesting experiment in participant-driven feature design.
	Specifically, the participants we are interested in driving the design of OASIS are experienced in architecture and modeling software as opposed to novice users.
	Although feedback from novice users is helpful in evaluating some user interface elements of OASIS, we require feedback from experienced users to make design choices pertaining to architectural analysis.       
	There are other applications where novice users feedback is readily required.
	User studies looking to gain generalized knowledge on human behavior require participation from novice users;
	for example a user study aimed at understanding what lines people most trace when trying to draw a human face would require a large novice user base to make statistically significant claims.\cite{limpaecher2013real}.
	Overall, the large number of users willing to try OASIS is promising because it will help us better guide future work.

	Interestingly, despite taking the time to register with OASIS, 26 participants ceased using our tool immediately and only 14 participants readily provided quantitative feedback for our pilot user study.
	The large number of users that ceased using OASIS immediately after registration is alarming;
	We are reliant on participants to help evaluate and guide feature design on OASIS.
	Having so many participants register but not use our tool, or readily provide feedback is a setback to the continual development of OASIS.
	Our hope is that OASIS will eventually attract a small but dedicated amount of experienced participants who will readily provide feedback upon updates to OASIS.
	Again, many of the users who register for OASIS and quickly leave might not be part of the small but dedicated participants we wish to appeal to;
	nonetheless if the reason users left OASIS is anything but disinterest in architectural sketching, we could potentially be driving away prospective dedicated participants.

	Another finding from our pilot study was that most of our participants had little to no experience with either architecture or the visual arts.	
	As mentioned previously, for some applications novice users are of particular interest; 
	however, for OASIS we are interested in experienced participants, who  ideally have used similar software, to provide informed feedback on design choices made for specialized features in OASIS.
	Novice users are helpful in diagnosing basic usability features, however they lack the specialized knowledge to drive the development of certain features.


	We also observed ambiguity in how participants answered feedback questions. 
	Currently, on OASIS there is no default answer for feedback questions and all feedback is provided voluntarily.
	This setup resulted in participants answering some feedback questions and skipping other questions.
	At the moment some questions are worded in a fashion that skipping them could imply feedback.
	Ambiguity in feedback from participants could lead us to make bad design choices on OASIS.
	These bad choices based on ambiguous feedback will cost us, the researcher, both time and effort.
	Rewording these questions and restructuring of how we collect feedback is recommended for future work if we wish to drive feature design on OASIS based on participant feedback.


	% inefficiently in the FGM, could lead to bad design choices



	Our study also showed a demand for doors and additional furniture items in our sketching interface.

	We also discovered that participants spent about 39\% of their time on the sketching interface and only 7\% of their time viewing daylight results and visualizations.
	
% ================================================
% significance/implications of all of our findings
% ================================================

	% when OASIS is put online, people will use it all the way through
	  % Introduce FGM
	  	% Introduce the two kinds of drivers, specialized vs general
		% Promising for guided research via FGM

	% many users register, few use our tool
	  % crucial if your FGM requires general users
	  % important because you want lure in vested users (he/she never tries cant become vest)

	% few users provide feedback relative to the users that actually use our tool
		% crucial if your FGM requires general users
		% Vested users == users who will return + give feedback. qed they would not be in this sit

	% advertisement on online forms with voting systems is difficult 
	  % crucial if your FGM requires general users
	  % crucial because you want to find your vested users

	% many novices few experts
		% positive result if your FGM requires general users
		% negative if your results require vested users

	% ambiguous feedback is hard to analyze
		% inefficiently in the FGM, could lead to bad design choices

	% strong demand for doors(arch)
		% - Doors == Multiroom
		% If you give people a sketching interface without doors, aparently they think you can't draw rooms anymore... thats just plain old neat! People are broken!

	% not enough people spend time on last tab (dyl)
		% (why) Might be having diffuiltiy getting to last tab
		% (why) Lose intrest along the way, no goal
		% (why) Models don't work
		% (why) Not understand what to do at tab, 
		% (expectations) maybe our visualizations are quick enough that poeople don't need too much time to sit back and appriciate...
			% Models = homework programs
			% Daylight issues = compilations error
			% you wouldn't spend hours looking at error, you'd just back to your code once you read the error.
		% > depends on the why in this case

	% furniture selection is limited (dyl)

% =============================================
% system Limitations and Future Work
% =============================================

	% Feedback Generation machine
	% ----------------------------------
		% Not enough users to make statistically significant claim
		% Timespan of pilot study (wish I had more time)
		% Requirement imposed by WebGL (graphics card)
		% Record more user actions on FGM
	   		% Timer system could have been better
			% which advertisement source they came from
			% what browser they have
			% mouse clicks
			% tools used on ui
			% if they consulted any help docs
		% Better handle ambiguous answers 
			% Convert some from active to passive
			% Make some mandatory to continue
		% Specific to OASIS
			% how they interacted with 3d model
				% maybe they did not even move it.
				% maybe they tried to do FPPOV view
			% Ask why they made renovations 
			    % b/c of us
			    % for daylighting changes

	% Architectural Sketching
	% ----------------------------------
		% Primitives Control 
			% wall control
				% No support for curved walls
				% Cannot define color of walls
				% cannot define Wall reflectance 
				% Free form lines for walls
				% no way to reposition a line after being drawn
				% No overdraw to redefine a line
			% window control
				% no control over window dimensions / exact placement
				% No varying types of glass user interface
				% No blinds on windows or shading devices
			% furniture
				% Requires more furniture, only bedroom right now
				% No control over furniture dimensions
				% No control over stacking of furniture
			% No support of doors 
			% No scale


		% Limitations inherited from using physical sketching algorithm
			% Offer only flat roofs 
			% Fails ambiguous sketches sometimes

	% Daylighting 
	% ----------------------------------

		% Time saver for users(can, but hard 2 do)
			% could generate multiple days and times we might be interested in making
			% No easy way to compare daylighting results 

		% Additional information we can provide given what we have
			% No sun paths
			% No north arrow
	   		% No control of thresholds for daylighting 

	   	% Additional information we would have to work hard to include
			% No other important values such as daylight factor 
			% No glare is taken into consideration
	    	% No varying types of glass for calculations
	    	% No support for blinds (yes sup. for shading devices)
	    	% No interior lighting 

		% Limitations inherited from using the daylight rendering engine
			% self occlusion of L building

	% OASIS Specific 	
	% ----------------------------------
		% No helpful errors
		% can only go only edit a model's latest renovation
		% currently non collaborative

	% Daylighting + Sketching Changes
	% ----------------------------------
		% outside building not taken into consideration 
		% No interior lighting 

Feedback from a pilot user study, which spanned only 2 weeks, shows OASIS was available to a broad range of users.
This is important because it shows that users studies can be conducted via OASIS to perform feature evaluations, A/B testing, and general surveys with a minimal cost of effort per user included in the study.  
There are currently no other online architectural sketching interfaces for simulations available; therefore,  gathering user feedback is important to the research of future daylighting visualizations and future improvements to our architectural sketching interface. 
In that respect, OASIS serves as a platform for sound research that can be easily guided by constant user feedback. 
Despite the fact that that the overall number of models produced on OASIS is less then number of models produced in previous studies on the Virtual Heliodon, the number of models we have collected so far is promising given the short span of time and small  audience we advertised to. 
Overall, the pilot study helped us better understand what to improve upon in OASIS.




% Similarly, there was no feedback concerning problems with accessing our online sketching interface.
% User feedback from our pilot user study overall positive feedback.
% However, it is also important to note that our pool of participants consist mostly of non-experts. 
% Whether our sketching interface was easy to use for experts remains unknown.
% Lastly, most feedback collected from participants concerning the effectiveness of OASIS for daylighting analysis was positive -- with over 90\% of participants stating that OASIS was effective and clear.
% Conversely, one participant with 10+ years job experience in architecture stated that OASIS was not useful for daylight analysis.
% This participant, however, was not clear as to why OASIS was not effective for daylighting analysis.
% Before any conclusion can be drawn, I believe we need to collect more feedback from experts and advertise our online tool towards users with architectural experience.
% In brief my contributions include the creation of a novel architectural sketching interface for simulations that is both accessible and easy-to-use for non-experts to perform qualitative daylighting analysis.
% In addition I conducted an online pilot user study and analyzed the results from that study.
% Moreover, I created an online framework for the physical sketch interpretation algorithm and daylight rendering engine used in the Virtual Heliodon.


\paragraph{Improved Evaluations of OASIS}
% My pilot user study was meant to be a short study to test key features of OASIS and understand problems users would encounter when using OASIS.
% Future user studies aimed at evaluating future iterations of OASIS in more detail can learn from mistakes made in this pilot user study.
% Firstly, the amount of users who registered and did not create a single model on OASIS is high.
% Despite lack of data on the retention rate of similar user studies, there are a few strategies that can be used to increase the number of users on OASIS.
% Goal driven user feedback may increase users interest in creating models on OASIS, sharing those models, and providing insightful feedback. 
% Disguising user studies as goal driven games has had successes in the crowd sourcing line drawings in previous research\cite{}.
% Having users use OASIS to fix a problem caused by daylighting, such glare in an office, with a scoring mechanism might incentive users to use OASIS as experts would. 
% Coupled with sharing features, that let users share 3D models or their scores would help OASIS self advertise itself to other users.
% Additional, making registration easier or even optional, for the first model, could increase user retention.
% Similarly, in our pilot study we encountered problems user made assumptions about feedback and models being viewed.
% Adding in a share feature, such as an automatically generated link that displays a model, would make sharing problem models much easier in feedback responses.
% Lastly, there was problem with users omitting feedback that could potentially communicate something to researchers.
% Making some feedback questions non-optional or inbreeding the feedback into the interface would make user intentions clearer.

\paragraph{Improvements To The Online Sketching Interface}
% Despite my original hypothesis not being met, the pilot user study did provide us feedback about what usability features to prioritize for future work.
% Firstly, the most requested feature in our sketching interface was support for doors.
% As mentioned previously, the physical sketch interpretation algorithm can interpret multi-room sketches.
% However, the lack of doors on the sketching interface communicated that we do not support multi-rooms sketches to many of our users.
% Additionally, the study revealed that room designs were greatly limited by the small number of furniture items in the system.
% While, OASIS is not meant to be a fully features daylighting analysis software, providing user a wider variety of furniture items would aide in the design of spaces other than bedrooms.
% Another common concern was scale, users wanted either control of scale in their sketches or more explicit communication of scale in our sketching interface.
% Giving users control of scale and providing an overlay of grid-lines on the sketching interface might better communicate scale of architectural spaces better than the currently indirect sense of scale through statically sized furniture items.

% Unrelated to results from our user study, OASIS is an online sketching interface for simulations. As of right now, the only true sketching features we support are straight walls. 
% Future work can be done to make our interface a full support sketching environment.
% Users in this sketching environment can draw not only straight walls, but also curved walls of any shape, in addition to sketching in windows and furniture items, similar to LightSketch\cite{}.
% Such a system would require some form of sketch recognition and a vocabulary or training sketches for common furniture items.
% Advantage so such a system would be one step closer to emulating how architects currently plan daylighting in the early design phase through rough pencil and paper sketches.

\paragraph{Improvements To Daylighting Visualizations}
% OASIS incorporates the daylight rendering engine from the Virtual Heliodon.
% The daylight rendering engine uses a GPU ray tracing framework, known as NVidia Optix\cite{}, to perform photon mapping at interactive rates.
% In the pipeline of the daylight rendering engine standard daylighting metrics such as the daylight factor, daylighting glare probability, and luminous flux per unit area, can be calculated and visualized in various ways.
% Future work can focus on creating informative daylight visualizations optimized for our online viewer.\\


